\section{Computation of Kronecker's Canonical Form}
The following chapter deals with the problem of computing Kronecker's canonical form for a matrix pair \((A, B)\).
The approach described will be shown to be correct, and an implementation
of it using the CAS SageMath has been made publicly available under MIT License on GitHub
\cite{Trapani_Computation_of_Kronecker_s}.

We shall divide it into two subproblems: the first one deals with regular pencils of matrices, the other with
singular pencils.

At the end of each of the following sections, the steps described shall be summarised in pseudocode.

\subsection*{Regular pencils.}
Let \(\Gamma(\lambda) = A + \lambda B\) be a regular pencil of matrices defined in a vector space over a
field \(F\).

First, let us clarify what we are to prove.
\begin{theorem}[KCF of a regular pencil of matrices] \cite[p. 28, Theorem 3]{gantmacher}
    Every regular pencil of matrices \(\Gamma(\lambda) = A + \lambda B\) can be reduced to a matrix of the form
    \[
        \begin{bmatrix}
            N^{(u_{1})} \\
            & N^{(u_{2})} \\
            & & \ddots \\
            & & & N^{(u_{s})} \\
            & & & & G + \lambda I
        \end{bmatrix},
    \]
    where the first \(s\) diagonal blocks correspond to infinite elementary divisors
    \(\mu^{u_{1}}, ..., \mu^{u_{s}}\) of \(\Gamma(\lambda)\) and the last block is uniquely determined by the
    finite elementary divisors of the given pencil.
\end{theorem}

\begin{proof}
    The following proof will describe a procedure to reduce \(\Gamma(\lambda)\) to Kronecker's canonical form.

    First, we shall find a value \(c \in F\) such that \(det(\Gamma(c)) \neq 0\) and define the matrix 
    \[
        A_{1} = A + cB.
    \]

    Now, we rewrite \(\Gamma(\lambda)\) in terms of \(A_{1}\),
    \[
        \Gamma(\lambda) = A_{1} + (\lambda - c)B,
    \]
    and premultiply it by \(A_{1}^{-1}\)
    \begin{gather}
        A_{1}^{-1} \Gamma(\lambda) = I + (\lambda - c)A_{1}^{-1}B. \label{gamma-reg-1}
    \end{gather}

    Let us denote with \(J\) the JCF of \(A_{1}^{-1}B\) and \(P_{1}\) the similarity matrix used to compute it
    \[
        J = P_{1}^{-1}(A_{1}^{-1}B)P_{1},
    \]
    and let us assume it is of the form
    \[
        J =
        \begin{bmatrix}
            J_{1} & \\
            & J_{0}
        \end{bmatrix}
    \]
    with \(J_{0}\) a nilpotent Jordan matrix and \(J_{1}\) such that \(det(J_{1}) \neq 0\) (which is the form returned by
    SageMath when computing Jordan matrices); call \(j\), \(k\) the number of rows in \(J_{0}\) and \(J_{1}\) respectively.

    Define the permutation matrix
    \[
        P_{\pi} =
        \begin{bmatrix}
            & I^{(k)}\\
            I^{(j)} &
        \end{bmatrix}
    \]
    so that we can write
    \[
        P_{\pi}^T J P_{\pi} = J' =
        \begin{bmatrix}
            J_{0} & \\
            & J_{1}
        \end{bmatrix}.
    \]

    Now, we rewrite \eqref{gamma-reg-1} substituting \(A_{1}^{-1}B\) with its permuted Jordan form
    \begin{gather}
        A_{1}^{-1} \Gamma(\lambda) = I + (\lambda - c)P_{1} P_{\pi}^{-T} J' P_{\pi}^{-1} P_{1}^{-1}.
        \label{gamma-reg-1.5}
    \end{gather}
    We can write the identity matrix in terms of \(P_{1}\) as \(I = P_{1}P_{1}^{-1}\) in \eqref{gamma-reg-1.5}
    \begin{align*}
        & A_{1}^{-1} \Gamma(\lambda) = \\
        &= P_{1}P_{1}^{-1} + (\lambda - c)P_{1} P_{\pi}^{-T} J' P_{\pi}^{-1} P_{1}^{-1} = \\
        &= P_{1}(P_{1}^{-1} + (\lambda - c)P_{\pi}^{-T} J' P_{\pi}^{-1} P_{1}^{-1}).
    \end{align*}
    We premultiply it by \(P_{1}^{-1}\)
    \[
        P_{1}^{-1} A_{1}^{-1} \Gamma(\lambda) = P_{1}^{-1} + (\lambda - c)P_{\pi}^{-T} J' P_{\pi}^{-1} P_{1}^{-1}
    \]
    and then postmultiply by \(P_{1}\)
    \[
        P_{1}^{-1} A_{1}^{-1} \Gamma(\lambda) P_{1} = I + (\lambda - c)P_{\pi}^{-T} J' P_{\pi}^{-1}.
    \]

    The very same steps can be followed for \(P_{\pi}^{-T}\) and \(P_{\pi}^{-1}\).

    The final result is
    \begin{gather}
        P_{\pi}^T P_{1}^{-1} A_{1}^{-1} \Gamma(\lambda) P_{1} P_{\pi} = I + (\lambda - c)J'. \label{gamma-reg-2}
    \end{gather}

    Now, we may work on the blocks \(J_{0}\), \(J_{1}\) of \(J'\).

    For the following steps, it is useful to rewrite the expression on the right side of \eqref{gamma-reg-2} so that
    the form of its diagonal blocks is explicitly readable, meaning
    \[
        I + (\lambda - c)J' =
        \begin{bmatrix}
            I^{(j)} - cJ_{0} + \lambda J_{0} & \\
            & I^{(k)} - cJ_{1} + \lambda J_{1}
        \end{bmatrix}.
    \]

    Let us start from the first diagonal block
    \[
        I^{(j)} - cJ_{0} + \lambda J_{0}.
    \]
    First, we need to postmultiply it in
    \eqref{gamma-reg-2} by \(K = (I^{(j)} - cJ_{0})^{-1}\)
    \begin{gather*}
        \scalemath{0.92}{
            P_{\pi}^T P_{1}^{-1} A_{1}^{-1} \Gamma(\lambda) P_{1} P_{\pi}
            \begin{bmatrix}
                K& \\
                & I^{(k)}
            \end{bmatrix}
            = 
            \begin{bmatrix}
                I^{(j)} + \lambda K J_{0} & \\
                & I^{(k)} - cJ_{1} + \lambda J_{1}
            \end{bmatrix}.
        }
    \end{gather*}

    Let us denote with \(H^{(j)}\) the JCF of \(K J_{0}\), \(P_{2}\) the similarity matrix
    used to compute it, and \(N^{(j)} = I^{(j)} + \lambda H^{(j)}\)
    \[
        N^{(j)} = I^{(j)} + \lambda H^{(j)} = I^{(j)} + \lambda P_{2}^{-1} (K J_{0}) P_{2}.
    \]

    \begin{remark}
        H is an upper shift matrix as it is the JCF of a nilpotent matrix.
    \end{remark}

    Following the analogous steps to handle \(P_{2}\) and its inverse yields us
    \begin{gather}
        \scalemath{0.92}{
            \begin{bmatrix}
                P_{2}^{-1} & \\
                & I
            \end{bmatrix}
            P_{\pi}^T P_{1}^{-1} A_{1}^{-1} \Gamma(\lambda) P_{1} P_{\pi}
            \begin{bmatrix}
                KP_{2} & \\
                & I^{(k)}
            \end{bmatrix} =
            \begin{bmatrix}
                N^{(j)} & \\
                & I^{(k)} - cJ_{1} + \lambda J_{1} 
            \end{bmatrix}. \label{gamma-reg-3}
        }
    \end{gather}

    At this point, we can focus on the second diagonal block
    \[
        I^{(k)} - cJ_{1} + \lambda J_{1}.
    \]
    We postmultiply it in
    \eqref{gamma-reg-3} by \(J_{1}^{-1}\)
    \begin{gather}
        \scalemath{0.92}{
            \begin{bmatrix}
                P_{2}^{-1} & \\
                & I
            \end{bmatrix}
            P_{\pi}^T P_{1}^{-1} A_{1}^{-1} \Gamma(\lambda) P_{1} P_{\pi}
            \begin{bmatrix}
                KP_{2} & \\
                & J_{1}^{-1}
            \end{bmatrix} =
            \begin{bmatrix}
                N^{(j)} & \\
                & J_{1}^{-1} + (\lambda - c)I^{(k)} 
            \end{bmatrix}.
        }
    \end{gather}

    Let us denote with \(G\) the JCF of the constant term \(J_{1}^{-1} - cI^{(k)}\) and \(P_{3}\) the similarity matrix
    used to compute it
    \[
        J_{1}^{-1} - cI^{(k)} = P_{3}^{-1}GP_{3}.
    \]

    Again, we follow the same procedure to handle \(P_{3}\) and its inverse, thus obtaining our end result
    \begin{gather}
        \scalemath{0.975}{
            \begin{bmatrix}
                P_{2}^{-1} & \\
                & P_{3}^{-1}
            \end{bmatrix}
            P_{\pi}^T P_{1}^{-1} A_{1}^{-1} \Gamma(\lambda) P_{1} P_{\pi}
            \begin{bmatrix}
                KP_{2} & \\
                & J_{1}^{-1}P_{3}
            \end{bmatrix} =
            \begin{bmatrix}
                N^{(j)} & \\
                & G + \lambda I^{(k)}
            \end{bmatrix}. \label{gamma-reg}
        }
    \end{gather}
\end{proof}


To conclude, we shall present the aforementioned steps used in order to compute Kronecker's canonical form
\(K(\Gamma(\lambda))\) in pseudocode.
\pagebreak
\begin{algorithm}[!h]\label{alg:kcf-regular}
    \caption{Procedure to compute Kronecker's canonical form of a regular pencil}
    \KwData{$\Gamma(\lambda) = A + \lambda B$: regular pencil}
    \KwResult{$(P, Q), K$: KCF of the pencil of matrices $\Gamma(\lambda)$}
    \While{True}{
        $c \gets $ random value\;
        \If{$det(\Gamma(c)) \neq 0$}{
            \Continue
        }
        $A_{1} \gets A + c*B$\;
        $P_1, J \gets $ jordan($A_{1}^{-1}*B$)\;
        $\{J_{0}, J_{1}\} \gets$ submatrices($J$)\;
        \If{$det(J_1) = 0$}{
            \Break
        }
    }
    $j, k \gets $nrows($J_0$), nrows($J_1$)\;
    $M \gets (I^{(j)} - c*J_{0})^{-1}$\;
    $P_2, N \gets $ jordan($M * J_0$)\;
    $P_3, G \gets $ jordan($J_{1}^{-1} - c*I^{(k)}$)\;
    $K \gets $ diag($N, G +\lambda I$)\;
    $P \gets A_1 * P_1 * $ block diagonal matrix([$P_2, P_3$])\;
    $Q \gets P_1 * $ block diagonal matrix([$M * P_2, J_1^{-1} * P_3$])\;
    \Return{$(P, Q), K$}\;
\end{algorithm}
\pagebreak
\subsection*{Singular pencils.}

Let \(\Gamma(\lambda)\) be a singular pencil of \(m \times n\) matrices the entries of which lie in a field \(F\)
\[\Gamma(\lambda) = A + \lambda B.\]
The pencil being singular implies either \(r < m\) or \(r < n\) with \(r\) rank of the pencil in the field \(F(\lambda)\)
of rational functions; let us assume without loss of generality the
relation \(r < n\) holds: otherwise, we shall work with the transpose of the pencil of matrices
\(\Gamma^{T}(\lambda) = A^T + \lambda B^T\) in order to reduce the problem to the case we shall describe hereafter.

First, we shall find a polynomial \(x(\lambda)\) of minimal degree \(\epsilon\) in the (right) kernel
of \(\Gamma(\lambda)\)
\[
    x(\lambda) = x_{0} - \lambda x_{1} + \lambda^2 x_{2} + ... + (-1)^\epsilon \lambda^\epsilon x_{\epsilon}.
\]
In order to define a procedure to compute it, we consider the family of \((k+2) \times (k+1)\) matrices
\(M_{k}^{(A, B)}\) defined as
\begin{align*}
    M_{0}^{(A, B)} &=
        \begin{bmatrix}
            A \\
            B
        \end{bmatrix},
    & M_{1}^{(A, B)} &=
        \begin{bmatrix}
            A & 0 \\
            B & A \\
            0 & B
        \end{bmatrix},
    & M_{k}^{(A, B)} &=
        \begin{bmatrix}
            A & 0 & \hdots &    0   \\
            B & A &        & \vdots \\
            0 & B & \ddots & \\
            \vdots & \vdots & \ddots & A \\
            0      &    0   & \hdots & B
        \end{bmatrix}.
\end{align*}
The basis matrix of the (right) kernel of a matrix \(M_{k}\) - split into blocks of size \(k+1\) - identifies
a polynomial \(g(\lambda)\) of degree \(k\) which satisfies the equations
\begin{align*}
   Ag_{0} &= 0, &
   Bg_{0} - Ag_{1} &= 0, &
   & ..., &
   Bg_{k-1} - Ag_{k} &= 0, &
   Bg_{k} &= 0.
\end{align*}
Note that if there is no polynomial of degree \(k\) in the kernel of the pencil
\(\Gamma(\lambda)\), then \(ker(M_{k})\) is empty.

Iteratively building the succession of matrices \(\{M_{i}\}_{i \geq 0}\) and computing their kernels
yields us the polynomial \(x(\lambda)\), which is what we were looking for.

Depending on the value of \(\epsilon\), we shall distinguish two cases.
\pagebreak
\begin{cs}
    \case \(\epsilon > 0.\)

    We want to prove the following theorem.
    \begin{theorem}[Reduction theorem] \cite[p. 30, Theorem 4]{gantmacher} \label{thm:reduction-theorem}
        Given a pencil of matrices \(\Gamma(\lambda)\), if the polynomial of minimal degree \(\epsilon\)
        in the (right) kernel of \(\Gamma(\lambda)\) has degree \(\epsilon > 0\), then \(\Gamma(\lambda)\) is equivalent
        to a pencil of the form
        \[
            \begin{bmatrix}
                L_{\epsilon} & 0 \\
                0 & \widehat{A} + \lambda \widehat{B}
            \end{bmatrix}.
        \]
    \end{theorem}
    \begin{proof}
        We shall prove the theorem by defining a procedure to compute such a transformation.

        We know both the vectors in \(B_1, B_2\) are linearly independent \cite[pp. 30-31]{gantmacher}, with
        \begin{align*}
            B_{1} &= \{x_{0}, x_{1}, ..., x_{\epsilon}\}, &
            B_{2} &= \{Ax_{0}, Ax_{1}, ..., Ax_{\epsilon}\}.
        \end{align*}

        Denote with \(P, Q\) the matrices obtained by completing the vectors in \(B_{2}, B_{1}\) to a basis in
        \(F^m, F^n\) respectively; applying a
        change of basis to the pencil of matrices \(\Gamma(\lambda)\) gives us
        \[
            \tilde{\Gamma}(\lambda) = \tilde{A} + \lambda \tilde{B} =  P^{-1}AQ + \lambda P^{-1}BQ,
        \]
        with \(\tilde{A}\), \(\tilde{B}\) of the kind
        \begin{align*}
            \tilde{A} &=
                \begin{bNiceArray}{>{\strut}lll}[margin=3mm]
                    \Block[draw=red]{1-2}{}
                    O^{(\epsilon, 1)} & I^{(\epsilon)} & \text{ *} \\
                    &  & \widehat{A}
                \end{bNiceArray}, &
            \tilde{B} &=
                \begin{bNiceArray}{>{\strut}lll}[margin=3mm]
                    \Block[draw=red]{1-2}{}
                    I^{(\epsilon)} & O^{(\epsilon, 1)} & \text{ *} \\
                    & & \widehat{B}
                \end{bNiceArray};
        \end{align*}
        the notation \(*\) is used for a block with no relevant structure.

        For the sake of clarity, we shall partition the pencil of matrices \(\tilde{\Gamma}(\lambda)\) as
        \[
            \begin{bmatrix}
                L_{\epsilon} & D + \lambda F \\
                0 & \widehat{A} + \lambda \widehat{B}
            \end{bmatrix}
        \]
        with \(L_{\epsilon}\) the first rectangular \(\epsilon \times (\epsilon + 1)\) block.

        Now we need to transform the pencil of matrices \(\tilde{\Gamma}(\lambda)\) so that it is upper triangular.

        We shall do this via the transformation
        \begin{equation} \label{singular-transformation}
            \begin{bmatrix}
                I^{(\epsilon)} & Y \\
                0 & I^{(m - \epsilon)}
            \end{bmatrix}
            \begin{bmatrix}
                L_{\epsilon} & D + \lambda F \\
                0 & \widehat{A} + \lambda \widehat{B}
            \end{bmatrix}
            \begin{bmatrix}
                I^{(\epsilon + 1)} & -X \\
                0 & I^{(n - \epsilon - 1)}
            \end{bmatrix}.
        \end{equation}
        Applying the transformation in \eqref{singular-transformation} yields
        \[
            \begin{bmatrix}
                L_{\epsilon} & D + \lambda F + Y(\widehat{A} + \lambda \widehat{B}) - L_{\epsilon}X \\
                0 & \widehat{A} + \lambda \widehat{B}
            \end{bmatrix}.
        \]
        In other words, to make the pencil \(\tilde{\Gamma}(\lambda)\) upper triangular, we shall determine
        two matrices \(X, Y\) such that
        \[
            D + \lambda F + Y(\widehat{A} + \lambda \widehat{B}) - L_{\epsilon}X = 0.
        \]
        Denoting with \(\vb{y}_{i}\), \(\vb{a}_{i}\), \(\vb{b}_{i}\) the i-th row of \(Y\), column of \(A\) and
        \(B\) respectively,
        we shall rewrite this condition as a system of linear equations
        \begin{equation} \label{linear-system-x}
            \begin{aligned}
            & \left\{
                \begin{aligned}
                    & x_{2, k} + \lambda x_{1, k} = d_{1, k} + \lambda f_{1, k} + \vb{y}_{1}\vb{a}_{k} +
                        \lambda \vb{y}_{1}\vb{b}_{k} \\
                    & x_{3, k} + \lambda x_{2, k} = d_{2, k} + \lambda f_{2, k} + \vb{y}_{2}\vb{a}_{k} +
                        \lambda \vb{y}_{2}\vb{b}_{k} \\
                    & \hdots \\
                    & x_{\epsilon+1, k} + \lambda x_{\epsilon, k} =
                        d_{\epsilon, k} + \lambda f_{\epsilon, k} + \vb{y}_{\epsilon}\vb{a}_{k} +
                        \lambda \vb{y}_{\epsilon}\vb{b}_{k} \\
                \end{aligned}
            \right. \\
            & & (k = 1, ..., n - \epsilon - 1)
        \end{aligned}
        \end{equation}
        To approach this problem, we shall define the \(1 \times (\epsilon - 1)(n - \epsilon - 1)\) matrix \(W\) as the differences between
        subsequent rows in \(D\) and \(F\) flipping the sign every \(m - \epsilon - 1\) elements, but first let us
        differentiate between two other subcases.
        
        To make this definition easier to understand, we shall present the pseudocode of an algorithm to compute \(W\).
        Now, we shall distinguish two other subcases.
        \subcase \(\epsilon = 1.\)

            In this case, there are no conditions on the values of the elements of \(Y\); we shall choose arbitrary values
            for its entries and compute those in \(X\) according to the last equation in the linear
            system in \eqref{linear-system-x}.
        \subcase \(\epsilon > 1.\)

        In this case, we shall define the matrix \(W\) according to the definition given above; in an attempt to make
        the explanation easier to understand, we shall give its definition (again, but) in pseudocode.

        \begin{algorithm*}
            \caption{Procedure to compute W.}
            $sign \gets -1$\;
            $W \gets []$\;
            $i \gets 0$\;
            \For(){$i < n - \epsilon - 1$}{
                $j \gets 0$\;
                \For(){$j < m - \epsilon - 1$}{
                    \If{$j \bmod (m - \epsilon - 1) = 0$}{
                        $sign \gets sign * (-1)$\;
                    }
                    append($W$, $sign * (f_{i+1, j} - d_{i, j})$)\;
                }
            }
            \Return{matrix($W$)}
        \end{algorithm*}

        Solving for \(Z\) in \(ZM_{\epsilon-2}^{(\widehat{A}, \widehat{B})} = W\) yields a matrix of the form
        (a reader shall refer to \cite[p.34]{gantmacher} for the proof of the solvability of the equation and 
        an explanation about the form of \(Z\))
        \[
            Z^T =
            \begin{bmatrix}
                \vb{y}_{1} - \vb{y}_{2} \\
                - (\vb{y}_{2} - \vb{y}_{3}) \\
                \hdots \\
                \vb{y}_{\epsilon-1} - \vb{y}_{\epsilon}
            \end{bmatrix}.
        \]
        We can now recover \(Y\) and, subsequently, \(X\) from the linear system in \eqref{linear-system-x}.
    \end{proof}

    To help make the procedure more intelligible, we shall present the pseudocode to apply the transformation in
    \nameref{thm:reduction-theorem} (theorem \ref{thm:reduction-theorem}).

    \pagebreak
    \begin{algorithm}[!h]\label{alg:kcf-singular}
        \caption{Procedure to reduce a singular pencil}
        \KwData{$\Gamma(\lambda) = A + \lambda B$: singular pencil}
        \KwResult{$K$: Reduced form of $\Gamma(\lambda)$}
        $V \gets $ polynomial of minimal degree $\in ker(\Gamma)$\;
        $Q \gets $ complete to a basis($V$)\;
        $P \gets $ complete to a basis($AV$)\;
        $\tilde{A} = P^{-1}AQ$\;
        $\tilde{B} = P^{-1}BQ$\;
        $L_{\epsilon}, D, F, \widehat{A}, \widehat{B} \gets $ partition pencil($\tilde{A} + \lambda \tilde{B}$)\;
        \eIf{$degree(V) \neq 0$}{
            $M \gets $ build $M_{\epsilon}(\widehat{A}, \widehat{B}, \epsilon-1)$\;
            $W \gets $ build $W(F, D)$\;
            $Z \gets WM^{-1}$\;
            $Y \gets $ partition(Z)\;
            $X \gets $ solve system in \eqref{linear-system-x}\;
            $L \gets $ block matrix($[[I, Y], [0, I]]$)\;
            $R \gets$ block matrix($[[I, -X], [0, I]]$)\;
            \Return{$LP^{-1}\Gamma(\lambda)QR$}\;
        }
        {\Return{$P^{-1}\Gamma(\lambda)Q$}\;}
    \end{algorithm}
    \case \(\epsilon = 0.\)

    In this case, the columns of the pencil are connected by linear relations with constant coefficients.

    This implies the transformation given by matrices \(P, Q\) returns a pencil of the form
    \[
        \begin{bmatrix}
            0 & A_{0} + B_{0}
        \end{bmatrix}.
    \]
    Assuming there are \(g\) constant independent solutions, then the pencil can be reduced to
    \[
        \begin{bmatrix}
            O^{(1, g)} & A_1 + \lambda B_1
        \end{bmatrix}.
    \]

    Supposing there are \(h\) constant independent solutions in the equation for \(\Gamma(\lambda)^T = 0\), then the first
    \(h\) rows can be reduced to zeros so that we obtain
    \[
        \widehat{\Gamma}(\lambda) = \begin{bmatrix}
            O^{(h, g)} \\
            & \widehat{A} + \lambda \widehat{B}
        \end{bmatrix}.
    \]
    Note there are no constant independent solutions in the pencil \(\widehat{A} + \lambda \widehat{B}\).
\end{cs}

We shall end our considerations on singular pencils by declaring the following lemma the proof of which can be
found in \cite[pp. 32-33]{gantmacher}.
\begin{lemma}
    If a pencil of matrices \(\Gamma(\lambda) = A + \lambda B\) has a polynomial of minimal degree \(\epsilon > 0\)
    in its (right) kernel, then the pencil \(\widehat{\Gamma}(\lambda)\)
    obtained via the transformation defined in \nameref{thm:reduction-theorem} has no polynomial in its (right)
    kernel of degree less than \(\epsilon\), with
    \[
        \widehat{\Gamma}(\lambda) = \widehat{A} + \lambda\widehat{B}.
    \]
\end{lemma}

Next, we shall show on the following page the pseudocode for an algorithm to compute Kronecker's canonical form of
an arbitrary pencil of matrices.

\pagebreak

\begin{algorithm}
    \SetKwBlock{Begin}{Begin}{}
    \SetKwFor{While}{while}{}{}
    \caption{Procedure to compute Kronecker's canonical form of an arbitrary pencil.}\label{alg:kcf}
    \KwData{$\Gamma(\lambda) = A + \lambda B$: pencil of \(m \times n\) matrices}
    \KwResult{$K$: Kronecker's canonical form of $\Gamma(\lambda)$}
    \Begin{
        \tcc{Edge case handling: A, B may be null matrices}
        \If{$A$ is zero and $B$ is zero}{
            \Return{$\Gamma(\lambda)$}\;
        }
        \tcc{L, R are the left and right transformation matrices respectively}
        $\tilde{\Gamma}(\lambda) \gets A + \lambda B$\;
        $L, R \gets I^{(m)}, I^{(n)}$\;
        \tcc{Toggled on when if the rightmost block on the bottom row has a left kernel}
        $transpose \gets False$\;
        \tcc{Repeat while the rightmost block on the bottom row is singular}
        \While{True}{
            \tcc{If the block is now regular}
            \If{dimensions($\tilde{A}$) = dimensions($\tilde{B}$) and
                det($\tilde{\Gamma}(\lambda) \neq 0$)}{
                    \Break\;
                }
            {
                \tcc{If the block has a left kernel}
                \If{$ncols(\tilde{A}) < nrows(\tilde{A}$)}
                {
                    $\tilde{\Gamma}(\lambda) \gets \tilde{\Gamma}^T(\lambda)$\;
                    $L, R, transpose \gets R^*, L^*, True$\;
                }
                $(P, Q), \tilde{\Gamma}(\lambda) \gets $ \Call Reduction theorem($\tilde{\Gamma}(\lambda)$)\;
                $u, w \gets$ nrows($P$), nrows($Q$)\;
                \tcc{Compose matrix transformations}
                $L \gets $ block diagonal matrix([$I^{(m - u)}, P$]) $ * L$\;
                $R \gets R * $ block diagonal matrix([$I^{(n - w)}, Q$])\;
                \tcc{$\tilde{\Gamma}(\lambda)$ is now the rightmost block on the bottom row}
                $\_, \tilde{\Gamma}(\lambda) \gets $ partition($\tilde{\Gamma}(\lambda)$)\;
            }
        }
    }
\end{algorithm}
\pagebreak
\begin{algorithm}
    \SetKwBlock{Begin}{}{end}
    \Begin{
        \Begin{
            \tcc{Next iteration, we shall check if there exists a right kernel first}
            \If{transpose}{
                $\tilde{\Gamma}(\lambda) \gets \tilde{\Gamma}^T(\lambda)$\;
                $L, R \gets R^*, L^*$\;
                $transpose \gets False$\;
            }
        }
        $P, Q \gets $ empty matrix, empty matrix\;
        \tcc{Check if $\tilde{\Gamma}(\lambda)$ is non-empty}
        \If{dimensions($\tilde{A}$) $\neq (0, 0)$}{
            $(P, Q), \_ \gets$  \Call KCF Regular Pencil($\tilde{\Gamma}(\lambda)$)\;
        }
        $u \gets$ nrows($P$)\;
        \tcc{Compose matrix transformations}
        $L \gets $ block diagonal matrix([$I^{(m - u)}, P^{-1}$]) $ * L$\;
        $R \gets R * $ block diagonal matrix([$I^{(n - u)}, Q$])\;
        \tcc{Return KCF of $\Gamma(\lambda)$}
        \Return{$L*\Gamma(\lambda)*R$}
    }
\end{algorithm}