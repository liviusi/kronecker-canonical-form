\section{Computation of Kronecker's Canonical Form}
The following chapter deals with the problem of computing Kronecker's canonical form for a matrix pair \((A, B)\).
The approach described will be shown to be correct, and an implementation
of it using the CAS SageMath has been made publicly available under MIT License on GitHub
\cite{Trapani_Computation_of_Kronecker_s}.


We shall divide the procedure into two steps: the first one deals with regular pencils of matrices, the other with
singular pencils.

At the end of each of the following sections, the steps described shall be summarised in pseudocode.

\subsection*{Regular pencils.}
Let \(\Gamma(\lambda) = A + \lambda B\) be a regular pencil of matrices defined in a vector space over a
field \(F\).

First, we shall find a value \(c \in F\) such that \(det(\Gamma(c)) \neq 0\) and define the matrix 
\[
    A_{1} = A + cB.
\]

Now, we rewrite \(\Gamma(\lambda)\) in terms of \(A_{1}\),
\[
    \Gamma(\lambda) = A_{1} + (\lambda - c)B,
\]
and premultiply it by \(A_{1}^{-1}\)
\begin{gather}
    A_{1}^{-1} \Gamma(\lambda) = I + (\lambda - c)A_{1}^{-1}B. \label{gamma-reg-1}
\end{gather}

Let us denote with \(J\) the JCF of \(A_{1}^{-1}B\) and \(P_{1}\) the similarity matrix used to compute it
\[
    J = P_{1}^{-1}(A_{1}^{-1}B)P_{1},
\]
and let us assume it is of the form
\[
    J =
    \begin{bmatrix}
        J_{1} & \\
        & J_{0}
    \end{bmatrix}
\]
with \(J_{0}\) a nilpotent Jordan matrix and \(J_{1}\) such that \(det(J_{1}) \neq 0\) (which is the form returned by
SageMath when computing Jordan matrices); call \(j\), \(k\) the number of rows in \(J_{0}\) and \(J_{1}\) respectively.

Define the permutation matrix
\[
    P_{\pi} =
    \begin{bmatrix}
        & I^{(k)}\\
        I^{(j)} &
    \end{bmatrix}
\]
so that we can write
\[
    P_{\pi}^T J P_{\pi} = J' =
    \begin{bmatrix}
        J_{0} & \\
        & J_{1}
    \end{bmatrix}.
\]

Now, we rewrite \eqref{gamma-reg-1} substituting \(A_{1}^{-1}B\) with its permuted Jordan form
\begin{gather}
    A_{1}^{-1} \Gamma(\lambda) = I + (\lambda - c)P_{1} P_{\pi}^{-T} J' P_{\pi}^{-1} P_{1}^{-1}.
    \label{gamma-reg-1.5}
\end{gather}
We can write the identity matrix in terms of \(P_{1}\) as \(I = P_{1}P_{1}^{-1}\) in \eqref{gamma-reg-1.5}
\begin{align*}
    & A_{1}^{-1} \Gamma(\lambda) = \\
    &= P_{1}P_{1}^{-1} + (\lambda - c)P_{1} P_{\pi}^{-T} J' P_{\pi}^{-1} P_{1}^{-1} = \\
    &= P_{1}(P_{1}^{-1} + (\lambda - c)P_{\pi}^{-T} J' P_{\pi}^{-1} P_{1}^{-1}).
\end{align*}
We premultiply it by \(P_{1}^{-1}\)
\[
    P_{1}^{-1} A_{1}^{-1} \Gamma(\lambda) = P_{1}^{-1} + (\lambda - c)P_{\pi}^{-T} J' P_{\pi}^{-1} P_{1}^{-1}
\]
and then postmultiply by \(P_{1}\)
\[
    P_{1}^{-1} A_{1}^{-1} \Gamma(\lambda) P_{1} = I + (\lambda - c)P_{\pi}^{-T} J' P_{\pi}^{-1}.
\]

The very same steps can be followed for \(P_{\pi}^{-T}\) and \(P_{\pi}^{-1}\).

The final result is
\begin{gather}
    P_{\pi}^T P_{1}^{-1} A_{1}^{-1} \Gamma(\lambda) P_{1} P_{\pi} = I + (\lambda - c)J'. \label{gamma-reg-2}
\end{gather}

Now, we may work on the blocks \(J_{0}\), \(J_{1}\) of \(J'\).

For the following steps, it is useful to rewrite the expression on the right side of \eqref{gamma-reg-2} so that
the form of its diagonal blocks is explicitly readable, meaning
\[
    I + (\lambda - c)J' =
    \begin{bmatrix}
        I^{(j)} - cJ_{0} + \lambda J_{0} & \\
        & I^{(k)} - cJ_{1} + \lambda J_{1}
    \end{bmatrix}.
\]

Let us start from the first diagonal block
\[
    I^{(j)} - cJ_{0} + \lambda J_{0}.
\]
First, we need to postmultiply it in
\eqref{gamma-reg-2} by \(K = (I^{(j)} - cJ_{0})^{-1}\)
\begin{gather*}
    \scalemath{0.92}{
        P_{\pi}^T P_{1}^{-1} A_{1}^{-1} \Gamma(\lambda) P_{1} P_{\pi}
        \begin{bmatrix}
            K& \\
            & I^{(k)}
        \end{bmatrix}
        = 
        \begin{bmatrix}
            I^{(j)} + \lambda K J_{0} & \\
            & I^{(k)} - cJ_{1} + \lambda J_{1}
        \end{bmatrix}.
    }
\end{gather*}

Let us denote with \(H^{(j)}\) the JCF of \(K J_{0}\), \(P_{2}\) the similarity matrix
used to compute it and \(N^{(j)} = I^{(j)} + \lambda H^{(j)}\)
\[
    N^{(j)} = I^{(j)} + \lambda H^{(j)} = I^{(j)} + \lambda P_{2}^{-1} (K J_{0}) P_{2}.
\]

\begin{remark}
    H is an upper shift matrix as it is the JCF of a nilpotent matrix.
\end{remark}

Following the analogous steps to handle \(P_{2}\) and its inverse yields us
\begin{gather}
    \scalemath{0.92}{
        \begin{bmatrix}
            P_{2}^{-1} & \\
            & I
        \end{bmatrix}
        P_{\pi}^T P_{1}^{-1} A_{1}^{-1} \Gamma(\lambda) P_{1} P_{\pi}
        \begin{bmatrix}
            KP_{2} & \\
            & I^{(k)}
        \end{bmatrix} =
        \begin{bmatrix}
            N^{(j)} & \\
            & I^{(k)} - cJ_{1} + \lambda J_{1} 
        \end{bmatrix}. \label{gamma-reg-3}
    }
\end{gather}

At this point, we can focus on the second diagonal block
\[
    I^{(k)} - cJ_{1} + \lambda J_{1}.
\]
We postmultiply it in
\eqref{gamma-reg-3} by \(J_{1}^{-1}\)
\begin{gather}
    \scalemath{0.92}{
        \begin{bmatrix}
            P_{2}^{-1} & \\
            & I
        \end{bmatrix}
        P_{\pi}^T P_{1}^{-1} A_{1}^{-1} \Gamma(\lambda) P_{1} P_{\pi}
        \begin{bmatrix}
            KP_{2} & \\
            & J_{1}^{-1}
        \end{bmatrix} =
        \begin{bmatrix}
            N^{(j)} & \\
            & J_{1}^{-1} + (\lambda - c)I^{(k)} 
        \end{bmatrix}.
    }
\end{gather}

Let us denote with \(G\) the JCF of the constant term \(J_{1}^{-1} - cI^{(k)}\) and \(P_{3}\) the similarity matrix
used to compute it
\[
    J_{1}^{-1} - cI^{(k)} = P_{3}^{-1}GP_{3}.
\]

Again, we follow the same procedure to handle \(P_{3}\) and its inverse, thus obtaining our end result
\begin{gather}
    \scalemath{0.975}{
        \begin{bmatrix}
            P_{2}^{-1} & \\
            & P_{3}^{-1}
        \end{bmatrix}
        P_{\pi}^T P_{1}^{-1} A_{1}^{-1} \Gamma(\lambda) P_{1} P_{\pi}
        \begin{bmatrix}
            KP_{2} & \\
            & J_{1}^{-1}P_{3}
        \end{bmatrix} =
        \begin{bmatrix}
            N^{(j)} & \\
            & G + \lambda I^{(k)}
        \end{bmatrix}. \label{gamma-reg}
    }
\end{gather}

To conclude, we shall present the aforementioned steps used in order to compute Kronecker's canonical form
\(K(\Gamma(\lambda))\) in pseudocode.
\begin{algorithm}
    \caption{Procedure to compute KCF of a regular pencil.}\label{alg:kcf-regular}
    \KwData{$\Gamma(\lambda) = A + \lambda B$: regular pencil}
    \KwResult{$K(\Gamma(\lambda))$: KCF of the pencil of matrices $\Gamma(\lambda)$}
    \While{True}{
        $c \gets $ random value\;
        \eIf{$det(\Gamma(c)) = 0$}{
            \Continue
        }
        {
            $A_{1} \gets A + c*B$\;
            $J \gets $ jordan($A_{1}^{-1}*B$)\;
            $\{J_{0}, J_{1}\} \gets$ submatrices($J$)\;
            $\{N_{i}\}_{i \geq 0} \gets $ jordan($(I - c*J_{0})^{-1} * J_{0}$)\;
            $G \gets $ jordan($J_{1}^{-1} - c*I$)\;
            $K(\Gamma(\lambda)) \gets $ diag($\{N_{i}\}, G +\lambda I$)
        }
    }
\end{algorithm}

We have now proved the following theorem.
\begin{theorem}[KCF of a regular pencil of matrices]
    Every regular pencil of matrices \(\Gamma(\lambda) = A + \lambda B\) can be reduced to a matrix of the form
    \[
        \begin{bmatrix}
            N^{(u_{1})} \\
            & N^{(u_{2})} \\
            & & \ddots \\
            & & & N^{(u_{s})} \\
            & & & & J + \lambda I
        \end{bmatrix},
    \]
    where the first \(s\) diagonal blocks correspond to infinite elementary divisors
    \(\mu^{u_{1}}, ..., \mu^{u_{s}}\) of \(\Gamma(\lambda)\) and the last block is uniquely determined by the
    finite elementary divisors of the given pencil.
\end{theorem}

\begin{theorem}[Stability of KCF transformation for a regular pencil] \label{thm:stability-kcf-regular}
    The problem of computing Kronecker's canonical form for a regular pencil
    is ill-conditioned.
\end{theorem} 

We can see \nameref{thm:stability-kcf-regular} (theorem \ref{thm:stability-kcf-regular}) is
trivially true as it involves the computation of many JCFs.

Now, we'll give an example to show how this may influence the end result.

\begin{example}
    
\end{example}

\subsection*{Singular pencils.}
In order to handle singular pencils of matrices, we must first prove the following theorem.

\begin{theorem}
    Given a singular pencil of matrices \(\Gamma(\lambda) = A + \lambda B\) of dimensions \(m \times n\) and
    rank \(r < n\), the degree \(\epsilon\) of the polynomial
    \[
        \vb{x}(\lambda) = x_{0} - \lambda x_{1} + \lambda^2 x_{2} - ... + (-1)^{\epsilon} \lambda^\epsilon x_{\epsilon}
    \]
    with \(x_{\epsilon} \neq 0\) is the least value for which the sign \(<\) holds in
    the relation \(\rho_{k} \leq (k+1)n\) where \(\rho_{k}\) is the rank of a \(k \times k + 1\)
    matrix of the form
    \[
        \begin{bmatrix}
            A & 0 & \hdots &    0   \\
            B & A &        & \vdots \\
            0 & B & \ddots & \\
            \vdots & \vdots & \ddots & A \\
            0      &    0   & \hdots & B
        \end{bmatrix}.
    \]
\end{theorem}
\begin{proof}
    Since the columns of the pencil of matrices \(\Gamma(\lambda)\) are linearly dependent, the equation
    \[
        (A+\lambda B)\vb{x} = 0
    \]
    has a non-zero solution. We choose the solution \(\vb{x}(\lambda)\) of the least possible degree \(\epsilon\).

    We substitute \(\vb{x}(\lambda)\) in the equation and obtain
    \begin{align*}
        Ax_{0} &= 0, & Bx_{0} - Ax_{1} &= 0,
        & ...&, & Bx_{\epsilon-1} - Ax_{\epsilon} &= 0, &
        Bx_{\epsilon} &= 0,
    \end{align*}

    which can, in turn, be considered as a system of linear homogenous equations for the elements
    of the columns \(x_{0} - x_{1}, x_{2} ..., (-1)^\epsilon x_{\epsilon}\); we can write the coefficient
    matrix of the system as
    \[
        M_{\epsilon} =
        \begin{bmatrix}
            A & 0 & \hdots &    0   \\
            B & A &        & \vdots \\
            0 & B & \ddots & \\
            \vdots & \vdots & \ddots & A \\
            0      &    0   & \hdots & B
        \end{bmatrix}.
    \]
    We know by construction that the rank of the matrix \(M_{\epsilon}\) \(\rho_{\epsilon} < (\epsilon+1)n\)
    and (that), since \(\epsilon\) is minimal, the ranks \(\rho_{0}, \rho_{1}, ..., \rho_{\epsilon-1}\)
    of the matrices
    \begin{align*}
        M_{0} &=
        \begin{bmatrix}
            A \\
            B
        \end{bmatrix},
        & M_{1} &=
        \begin{bmatrix}
            A & 0 \\
            B & A \\
            0 & B
        \end{bmatrix},
        & M_{\epsilon-1} &=
        \begin{bmatrix}
            A & 0 & \hdots &    0   \\
            B & A &        & \vdots \\
            0 & B & \ddots & \\
            \vdots & \vdots & \ddots & A \\
            0      &    0   & \hdots & B
        \end{bmatrix}
    \end{align*}
    satisfy the equations \(\rho_{0} = n, \rho_{1} = 2, \rho_{\epsilon-1} = \epsilon n\) thus
    proving the theorem.
\end{proof}

At this point, we can state and prove a fundamental theorem which will certainly aid us
in proving \nameref{thm:kcf} (theorem \ref{thm:kcf}).

\begin{theorem}[Reduction theorem]
    If the equation given by a matrix pair \(\Gamma(\lambda) = (A, B)\) has a solution of minimal
    degree \(\epsilon > 0\), then \(\Gamma(\lambda)\) is strictly equivalent to a pencil of matrices of the form
    \[
        \begin{bmatrix}
            L_{\epsilon} & 0 \\
            0 & \tilde{A} + \lambda \tilde{B}
        \end{bmatrix},
    \]
    where the equation analogous for \((\tilde{A}, \tilde{B})\) has no solution of degree \(\alpha < \epsilon\) and
    \(L_{\epsilon}\) is an \(\epsilon \times (\epsilon + 1)\) matrix such that
    \[
        \begin{bmatrix}
            \lambda & 1       & 0     & \hdots & 0       &    0   \\
            0       & \lambda & 1     &        & \vdots  & \vdots \\
            \vdots  & \vdots & \ddots & \ddots \\
            0       &        &        &        & \lambda & 1
        \end{bmatrix}.
    \]
\end{theorem}

\begin{proof}
    
\end{proof}