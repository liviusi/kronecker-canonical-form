\section{Implementation}
The following chapter shall go into a few technical details on the implementation of the algorithm provided in
\cite{Trapani_Computation_of_Kronecker_s}.

\subsection*{Setup instructions.}
First, it is required to clone the repository.

In order to test the code provided or include it in scripts, it is required to follow the following steps.

Setup Python \cite{van1995python} on your machine; the version the algorithm has been implemented with
is \textbf{Python 3.10.6}.

Setup SageMath \cite{sage} on your machine and make sure Python can handle SageMath code, this should be achieved
by installing the Python package \textit{sagemath}; on success, the following snippet does not produce any
errors.
\begin{minted}[frame=lines, framesep=2mm]{python}
from sage.all import *
\end{minted}

Last, install \textbf{make} and the Python package \textbf{pytest}.

Next, a dump on the versions of the aforementioned utilities.
\begin{minted}[frame=lines, framesep=2mm, escapeinside=||, texcomments]{python}
|\textcolor{red}{\$}| python --version
Python 3.10.6
|\textcolor{red}{\$}| sage --version
SageMath version 9.6, Release Date: 2022-05-15
|\textcolor{red}{\$}| make --version
GNU Make 4.3
|\textcolor{red}{\$}| pytest --version
pytest 7.1.2
\end{minted}

\subsection*{How to run tests.}
The method chosen to test the code provided is unit testing, which pytest makes possible.
To run the test suite, it is required to execute the following command from the root of the repository.
\begin{minted}[frame=lines, framesep=2mm, escapeinside=||, texcomments]{python}
|\textcolor{red}{\$}| make tests
\end{minted}

It is also possible to get a report on the test coverage provided by the test suite.
\begin{minted}[frame=lines, framesep=2mm, escapeinside=||, texcomments]{python}
|\textcolor{red}{\$}| coverage run -m pytest -v tests && coverage report -m
\end{minted}

The output indicates every line of code has been run, except for those used for debugging purposes and those
which do not give as an output the matrices used when computing a transformation.

\subsection*{How to use.}
Invoking the command \mintinline{bash}{sage} on a script named \mintinline{bash}{script.sage} produces a (pre-)parsed
Python module named \mintinline{bash}{script.sage.py} and subsequently runs it. The newly produced file can also
be run with \mintinline{bash}{python script.sage.py} as it is a valid Python module, but its name containing
\mintinline{bash}{.sage.py} makes it unusable from either SageMath or Python as an external module. This means the
following code produces errors.
\begin{minted}[frame=lines, framesep=2mm]{python}
from script.sage import *
\end{minted}

This problem can be solved by renaming the resulting file, as in the \textbf{Makefile} provided; thus, the first step to
use the provided code is to run
\begin{minted}[frame=lines, framesep=2mm, escapeinside=||, texcomments]{python}
|\textcolor{red}{\$}| make
\end{minted}

At this point, it is possible to include the functions defined either in a Python or a SageMath environment.
\begin{minted}[frame=lines, framesep=2mm]{python}
from kcf_sage import kcf
\end{minted}

\subsection*{Time complexity.}
As the algorithm works with computer algebra, an actual estimation of the time complexity of elementary operations
(such as the sum of two variables) is implementation-dependent; what we can do is estimate the number of operations
needed at each step and give a measure of the complexity of the whole procedure based on it.

Assume the starting point is an arbitrary pencil of \(m \times n\) matrices \(\Gamma(\lambda) = A+\lambda B\).

The method \mintinline{bash}{_reduction_theorem}, which operates on singular pencils of matrices,
is called \(p + q\) times, with \(p\) the number of \(L_\epsilon\) blocks and \(q\)
that of \(L^T_\eta\) blocks; internally, it calls \mintinline{bash}{_compute_lowest_degree_polynomial}, which
iteratively computes the right kernel of an \(M_i\) matrix with \(i = 0, ..., k\) and \(k\) is the minimal degree
of the polynomial at this point.

\begin{remark}
    The time complexity for Gaussian elimination on an \(m \times n\) matrix is
    \(\Theta(mn \min(m, n))\) \cite{boyd_vandenberghe_2018}.
\end{remark}

To summarize, the number of operations needed to reduce an arbitrary singular pencil of matrices is
\[
    \mathcal{O}((p + q)\alpha mn \min(m, n)) = \mathcal{O}(\alpha mn \min(m, n)^2),
\]
with \(\alpha = \max(\{\epsilon_0, ..., \epsilon_p, \eta_0, ..., \eta_q\})\).

Next, let \(\Gamma(\lambda) = A + \lambda B\) be a regular pencil of \(n \times n\) matrices. To make calculations
easier, we shall assume the factorization of the characteristic polynomial of \(\Gamma(\lambda)\) is known.

\begin{remark}
    Given the factorization of the characteristic polynomial of an \(n \times n\) matrix, the time complexity
    for the algorithm to compute its Jordan canonical form is \(\mathcal{O}(n^4)\)
    \cite{DBLP:journals/corr/abs-cs-0412005}.
\end{remark}

We can conclude the number of operations for a regular matrix pair is
\[
    \mathcal{O}(3n^4) = \mathcal{O}(n^4).
\]
Tying all together, the number of operations needed to compute Kronecker's canonical form is
\[
    \mathcal{O}(\alpha mn \min(m, n)^2 + (\max(m, n) - p - q)^4).
\]

\begin{remark}
    To make calculations feasible, we shall expect the \(n' \times n'\) regular block in an arbitrary pencil of
    \(m \times n\) matrices not to be bigger than a \(4 \times 4\) matrix as it is necessary to factor an
    \(n'\)-degree polynomial, which is not possible via symbolic computations in the general case for polynomials
    of degree \(5\) or more. This means the number of operations the algorithm needs is typically no more than
    \[
        256 + \alpha mn \min(m, n)^2.
    \]
\end{remark}

\subsection*{Minimal working example.}
Pencils of matrices are chosen to be represented as two variables. The snippet provided on the following page
shall explain how to call the functions defined in order to compute Kronecker's canonical form of the pencil
\(\Gamma(\lambda) = A + \lambda B\) subjected to an arbitrary change of basis, with
\begin{align*}
    A &=
    \begin{bmatrix}
        0 & 1 & 0 & 0 \\
        0 & 0 & 1 & 0 \\
        0 & 0 & 0 & 1 \\
        & & & & 1 & 0 \\
        & & & & 0 & 1 \\
        & & & & & & 42
    \end{bmatrix} &
    B &=
    \begin{bmatrix}
        1 & 0 & 0 & 0 \\
        0 & 1 & 0 & 0 \\
        0 & 0 & 1 & 0 \\
        & & & & 0 & 1 \\
        & & & & 0 & 0 \\
        & & & & & & 1
    \end{bmatrix}.
\end{align*}
\pagebreak
\begin{minted}[numbersep=5pt,
    frame=lines,
    framesep=2mm]{python}
import sage.all as sa
from kcf import kcf_sage as kcf
L3_A = sa.matrix(sa.SR, [[0, 1, 0, 0],
      [0, 0, 1, 0], [0, 0, 0, 1]])
L3_B = sa.matrix(sa.SR, [[1, 0, 0, 0],
      [0, 1, 0, 0], [0, 0, 1, 0]])
A = sa.block_diagonal_matrix([L3_A,
      sa.identity_matrix(2),
      sa.matrix(sa.SR, [[42]])])
B = sa.block_diagonal_matrix([L3_B,
      sa.matrix(sa.SR, [[0, 1], [0, 0]]),
      sa.identity_matrix(1)])
while True:
    D = sa.random_matrix(sa.ZZ,
          A.nrows(), A.nrows()).change_ring(sa.SR)
    if not (D.det().is_zero()):
      while True:
          C = sa.random_matrix(sa.ZZ, A.ncols(),
                A.ncols()).change_ring(sa.SR)
          if not (C.is_zero()):
              break
      break
A = D.inverse() * A * C
B = D.inverse() * B * C
(L, R), (KCF_A, KCF_B) = kcf.kronecker_canonical_form(A, B, True)
assert ((L*A*R - KCF_A).is_zero() and not L.det().is_zero()
      and (L*B*R - KCF_B).is_zero() and not R.det().is_zero())
print(f'KCF:\n{kcf.stringify_pencil(KCF_A, KCF_B)}')
\end{minted}