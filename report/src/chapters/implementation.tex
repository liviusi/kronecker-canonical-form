\section{Implementation}
The following chapter shall go into a few technical details on the implementation of the algorithm provided in
\cite{Trapani_Computation_of_Kronecker_s}.

For the code to run on your local machine, it is required to setup \textbf{Python} \cite{van1995python},
\textbf{SageMath} and \textbf{Pytest}.

\subsection*{How to use.}
Invoking the command \mintinline{bash}{sage} on a script named \mintinline{bash}{script.sage} produces a (pre-)parsed
Python module named \mintinline{bash}{script.sage.py} and subsequently runs it. The newly produced file can also
be run with \mintinline{bash}{python script.sage.py} as it is a valid Python module, but its name containing
\mintinline{bash}{.sage.py} makes it unusable from either SageMath or Python as an external module. This means the
following code produces errors.
\begin{minted}[frame=lines, framesep=2mm]{python}
from script.sage import *
\end{minted}

This problem can be solved by renaming the resulting file, as in the \textbf{Makefile} provided.

At this point, it is possible to include the functions defined either in a Python or a SageMath environment.

The function implementing the algorithm defined in \nameref{alg:kcf} (algorithm \ref{alg:kcf}) has been designed
so that a SageMath developer may find its usage similar to that of \mintinline{bash}{jordan_form}: its signature
contains the optional boolean parameter \mintinline{bash}{transformation}; if explicitly toggled on, the function
shall return the transformation matrices too, as is the case with the method \mintinline{bash}{jordan_form}.

\subsection*{Testing the code.}
To ensure there are no errors in the implementation, a suitable
test suite has been provided. We've chosen to use a functional testing technique known as \textbf{unit testing},
and the test suite accurately tests the function \mintinline{bash}{kcf}.

We've defined matrices already in Kronecker's canonical form and applied a random change of basis in order to have
deterministic output from non-deterministic input so that we may assume we have covered every possible case.

There are three files populated with functions to build completely regular, singular and a mix of both
pencils of matrices, apply a random change of basis transformation, compute Kronecker's canonical form for each
of them and, last, assert the transformation given in output is correct.

The tool used for testing - \textit{pytest} - can also provide information on
the test coverage, which is 91\%.


\subsection*{Time complexity.}
As the algorithm works with computer algebra, an actual estimation of the time complexity of elementary operations
(such as the sum of two variables) is implementation, ring and value-dependent; what we can do is calculate
the number of operations needed at each step and give a measure of the complexity of the whole procedure based on it.

Assume the starting point is an arbitrary pencil of \(m \times n\) matrices \(\Gamma(\lambda) = A+\lambda B\).

The method \mintinline{bash}{_reduction_theorem}, which operates on singular pencils of matrices,
is called \(p + q\) times, with \(p\) the number of \(L_\epsilon\) blocks and \(q\)
that of \(L^T_\eta\) blocks; internally, it calls \mintinline{bash}{_compute_lowest_degree_polynomial}, which
iteratively computes the right kernel of an \(M_i\) matrix with \(i = 0, ..., k\) and \(k\) is the minimal degree
of the polynomial at this point.

\begin{remark}
    The time complexity for Gaussian elimination on an \(m \times n\) matrix is
    \(\Theta(mn \min(m, n))\) \cite{boyd_vandenberghe_2018}.
\end{remark}

To summarize, the number of operations needed to reduce an arbitrary singular pencil of matrices is
\[
    \mathcal{O}((p + q)\alpha mn \min(m, n)) = \mathcal{O}(\alpha mn \min(m, n)^2),
\]
with \(\alpha = \max(\{\epsilon_0, ..., \epsilon_p, \eta_0, ..., \eta_q\})\).

Next, let \(\Gamma(\lambda) = A + \lambda B\) be a regular pencil of \(n \times n\) matrices. To make calculations
easier, we shall assume the factorization of the characteristic polynomial of \(\Gamma(\lambda)\) is known.

\begin{remark}
    Given the factorization of the characteristic polynomial of an \(n \times n\) matrix, the time complexity
    for the algorithm to compute its Jordan canonical form is \(\mathcal{O}(n^4)\)
    \cite{DBLP:journals/corr/abs-cs-0412005}.
\end{remark}

We can conclude the number of operations for a regular matrix pair is
\[
    \mathcal{O}(3n^4) = \mathcal{O}(n^4).
\]
Tying all together, the number of operations needed to compute Kronecker's canonical form is
\[
    \mathcal{O}(\alpha mn \min(m, n)^2 + (\max(m, n) - p - q)^4).
\]

\subsection*{Minimal working example.}
Pencils of matrices are chosen to be represented as two variables. The snippet provided on the following page
shall explain how to call the functions defined in order to compute Kronecker's canonical form of the pencil
\(\Gamma(\lambda) = A + \lambda B\) subjected to an arbitrary change of basis, with
\begin{align*}
    A &=
    \begin{bmatrix}
        0 & 1 & 0 & 0 \\
        0 & 0 & 1 & 0 \\
        0 & 0 & 0 & 1 \\
        & & & & 1 & 0 \\
        & & & & 0 & 1 \\
        & & & & & & 42
    \end{bmatrix} &
    B &=
    \begin{bmatrix}
        1 & 0 & 0 & 0 \\
        0 & 1 & 0 & 0 \\
        0 & 0 & 1 & 0 \\
        & & & & 0 & 1 \\
        & & & & 0 & 0 \\
        & & & & & & 1
    \end{bmatrix}.
\end{align*}
\pagebreak
\begin{minted}[numbersep=5pt,
    frame=lines,
    framesep=2mm]{python}
import sage.all as sa
from kcf import kcf_sage as kcf


def random_invertible_matrix(n: int):
    while True:
        M = sa.random_matrix(sa.ZZ, n, n)
        if not (M.is_singular()):
            return M.change_ring(sa.SR)


L3_A = sa.matrix(sa.SR, [[0, 1, 0, 0],
                         [0, 0, 1, 0],
                         [0, 0, 0, 1]])
L3_B = sa.matrix(sa.SR, [[1, 0, 0, 0],
                         [0, 1, 0, 0],
                         [0, 0, 1, 0]])
A = sa.block_diagonal_matrix([L3_A,
                              sa.identity_matrix(2),
                              sa.matrix(sa.SR, [[42]])])
B = sa.block_diagonal_matrix([L3_B,
                              sa.matrix(sa.SR, [[0, 1],
                                                [0, 0]]),
                              sa.identity_matrix(1)])

D = random_invertible_matrix(A.nrows())
C = random_invertible_matrix(A.ncols())

A = D.inverse() * A * C
B = D.inverse() * B * C

(L, R), (KCF_A, KCF_B) = kcf.kronecker_canonical_form(
    A, B, transformation=True)
assert ((L*A*R - KCF_A).is_zero()
        and (L*B*R - KCF_B).is_zero()
        and not L.is_singular() and not R.is_singular())
print(f'KCF:\n{kcf.stringify_pencil(KCF_A, KCF_B)}')
\end{minted}