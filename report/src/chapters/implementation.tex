\section{Implementation}
The following chapter shall go into a few technical details on the implementation of the algorithm provided in
\cite{trapani-kronecker}.

For the code to run on your local machine, it is required to setup \textbf{Python} \cite{van1995python},
\textbf{SageMath} and \textbf{Pytest}.

\subsection*{How to use.}
Invoking the command \mintinline{bash}{sage} on a script named \mintinline{bash}{script.sage} produces a (pre-)parsed
Python module named \mintinline{bash}{script.sage.py} and subsequently runs it. The newly produced file can also
be run with \mintinline{bash}{python script.sage.py} as it is a valid Python module, but its name containing
\mintinline{bash}{.sage.py} makes it unusable from either SageMath or Python as an external module. This means the
following code produces errors.
\begin{minted}[frame=lines, framesep=2mm]{python}
from script.sage import *
\end{minted}

This problem can be solved by renaming the resulting file, as in the \textbf{Makefile} provided.

At this point, it is possible to include the functions defined either in a Python or a SageMath environment.

The function implementing the algorithm defined in algorithm \ref{alg:kcf} (\nameref{alg:kcf}) has been designed
so that a SageMath developer may find its usage similar to that of \mintinline{bash}{jordan_form}: its signature
contains the optional boolean parameter \mintinline{bash}{transformation}; if explicitly toggled on, the function
shall return the transformation matrices too, as is the case with the method \mintinline{bash}{jordan_form}.

\subsection*{Testing the code.}
To ensure there are no errors in the implementation, a suitable
test suite has been provided. We've chosen to use a functional testing technique known as \textbf{unit testing},
and the test suite accurately tests the function \mintinline{bash}{kcf}.

The test suite consists of three files populated with functions which build pencils of matrices either with
only regular blocks or only singular blocks or combinations of both, (already) in Kronecker's canonical form,
apply a random change of basis transformation, compute Kronecker's canonical form and assert the transformation
in output is correct. Since we are testing pencils of matrices made up of only regular blocks, only singular blocks
and a mix of both, and we are applying a non-deterministic transformation, by repeating the tests we may assume
we have covered as many cases as possible.

The tool used for testing - \textit{pytest} - can also provide information on
the test coverage, which is 91\%; the reasons it is not 100\% are two: first, there is a routine which, given a
pencil of matrices returns its string representation, and it is not called unless tests fail; second, the code
we're testing includes - in its internal routines (i.e. those with a leading underscore in their names) - some
lines to handle invalid input values, and the ``public'' routines do not use invalid input values.

On the following page, we provide the reader with a minimal working example of the usage of the code written. We
use two distinct variables for the (two) matrices making up the pencil of matrices.

The snippet works the very same way the functions in the test suite have been defined: starting with a pencil
of matrices in Kronecker's canonical form, we apply a random change of basis transformation and compute Kronecker's
canonical form; next, we assert the transformation given in output is correct. Last, we print the string
representation of Kronecker's canonical form of the pencil of matrices in input.

The pencil of matrices \(\Gamma(\lambda) = A + \lambda B\) the example is working with is

\begin{align*}
    A &=
    \begin{bmatrix}
        0 & 1 & 0 & 0 \\
        0 & 0 & 1 & 0 \\
        0 & 0 & 0 & 1 \\
        & & & & 1 & 0 \\
        & & & & 0 & 1 \\
        & & & & & & 42
    \end{bmatrix}, &
    B &=
    \begin{bmatrix}
        1 & 0 & 0 & 0 \\
        0 & 1 & 0 & 0 \\
        0 & 0 & 1 & 0 \\
        & & & & 0 & 1 \\
        & & & & 0 & 0 \\
        & & & & & & 1
    \end{bmatrix}.
\end{align*}

\pagebreak


\begin{minted}[numbersep=5pt,
    frame=lines,
    framesep=2mm]{python}
import sage.all as sa
from kcf import kcf_sage as kcf


def random_invertible_matrix(n: int):
    while True:
        M = sa.random_matrix(sa.ZZ, n, n)
        if not (M.is_singular()):
            return M.change_ring(sa.SR)


L3_A = sa.matrix(sa.SR, [[0, 1, 0, 0],
                         [0, 0, 1, 0],
                         [0, 0, 0, 1]])
L3_B = sa.matrix(sa.SR, [[1, 0, 0, 0],
                         [0, 1, 0, 0],
                         [0, 0, 1, 0]])
A = sa.block_diagonal_matrix([L3_A,
                              sa.identity_matrix(2),
                              sa.matrix(sa.SR, [[42]])])
B = sa.block_diagonal_matrix([L3_B,
                              sa.matrix(sa.SR, [[0, 1],
                                                [0, 0]]),
                              sa.identity_matrix(1)])

D = random_invertible_matrix(A.nrows())
C = random_invertible_matrix(A.ncols())

A = D.inverse() * A * C
B = D.inverse() * B * C

(L, R), (KCF_A, KCF_B) = kcf.kronecker_canonical_form(
    A, B, transformation=True)
assert ((L*A*R - KCF_A).is_zero()
        and (L*B*R - KCF_B).is_zero()
        and not L.is_singular() and not R.is_singular())
print(f'KCF:\n{kcf.stringify_pencil(KCF_A, KCF_B)}')
\end{minted}

\pagebreak


\subsection*{Time complexity.}
As the algorithm works with computer algebra, an actual estimation of the time complexity of elementary operations
(such as the sum of two variables) is implementation, ring and value-dependent; what we can do is calculate
the number of operations needed at each step and give a measure of the complexity of the whole procedure based on it.

Assume the starting point is an arbitrary pencil of \(m \times n\) matrices \(\Gamma(\lambda) = A+\lambda B\).

The method \mintinline{bash}{_reduction_theorem}, which operates on singular pencils of matrices,
is called \(p + q\) times, with \(p\) the number of \(L_\epsilon\) blocks and \(q\)
that of \(L^T_\eta\) blocks; internally, it calls \mintinline{bash}{_compute_lowest_degree_polynomial}, which
iteratively computes the right kernel of an \(M_i\) matrix with \(i = 0, ..., k\) and \(k\) is the minimal degree
of the polynomial at this point.

\begin{remark}
    The time complexity for Gaussian elimination on an \(m \times n\) matrix is
    \(\Theta(mn \min(m, n))\) \cite{boyd_vandenberghe_2018}.
\end{remark}

To summarize, the number of operations needed to reduce an arbitrary singular pencil of matrices is
\[
    \mathcal{O}((p + q)\alpha mn \min(m, n)) = \mathcal{O}(\alpha mn \min(m, n)^2),
\]
with \(\alpha = \max(\{\epsilon_0, ..., \epsilon_p, \eta_0, ..., \eta_q\})\).

Next, let \(\Gamma(\lambda) = A + \lambda B\) be a regular pencil of \(n \times n\) matrices. To make calculations
easier, we shall assume the factorization of the characteristic polynomial of \(\Gamma(\lambda)\) is known.

\begin{remark}
    Given the factorization of the characteristic polynomial of an \(n \times n\) matrix, the time complexity
    for the algorithm to compute Jordan's canonical form is \(\mathcal{O}(n^4)\)
    \cite{DBLP:journals/corr/abs-cs-0412005}.
\end{remark}

Since the number of Jordan matrices to be computed in the case of a regular pencil is - at most - 3, we can conclude
the number of operations for a regular matrix pair is
\[
    \mathcal{O}(3n^4) = \mathcal{O}(n^4).
\]
Tying all together, the number of operations needed to compute Kronecker's canonical form is
\[
    \mathcal{O}(\alpha mn \min(m, n)^2 + (\max(m, n) - p - q)^4).
\]