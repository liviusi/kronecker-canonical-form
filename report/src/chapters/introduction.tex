\section{Introduction}
Consider the linear differential equation
\[
	\dot{x} + Ax = f.
\]
The set of solutions of this equation is characterized by matrix $A$ and, in particular, by its Jordan's canonical form.

Next, take the equation
\begin{gather} \label{intro:1}
	B\dot{x} + Ax = f,
\end{gather}
with $A, B \in \mathbb{C}^{m \times n}$. We call the pair ($A$, $B$) a linear pencil of matrices or, for the sake of brevity,
a pencil of matrices \cite{ikramov}.

To solve \eqref{intro:1}, we shall distinguish two cases.
\begin{cs}
	\case $B$ is nonsingular.

		The set of solutions is characterized by Jordan's canonical form of \[-B^{-1}A.\]
	\case $B$ is singular.

		The linear system given by \eqref{intro:1} may contain algebraic equations or, in other words, equations which do
		not contain any derivatives. As an example, take the equation
		\[
			\begin{bmatrix}
				1 & 0 & 1 \\
				0 & 0 & 0
			\end{bmatrix}
			\begin{bmatrix}
				\dot{x_1} \\
				\dot{x_2} \\
				\dot{x_3}
			\end{bmatrix} + 
			\begin{bmatrix}
				1 & 0 & 0 \\
				1 & 1 & 1
			\end{bmatrix}
			\begin{bmatrix}
				x_1 \\
				x_2 \\
				x_3
			\end{bmatrix}
			= \begin{bmatrix}
				f_1 \\
				f_2
			\end{bmatrix}.
		\]
		The linear system contains an algebraic equation
		\begin{equation*}
			\left\{
				\begin{aligned}
					\dot{x_1} + \dot{x_3} + x_1 &= f_1 \\
					x_1 + x_2 + x_3 &= f_2
				\end{aligned}
			\right.
		\end{equation*}
		The set of solutions is characterized by \textbf{Kronecker's canonical form} of ($A$, $B$)
		\cite{gantmacher, kunkel-mehrmann}.
\end{cs}

The problem of computing Kronecker's canonical form is ill-conditioned, meaning that small changes in its input values cause
a large change in the answer. The current state of the art is to compute a \textbf{Kronecker-like form}, which can be found with
numerically stable algorithms \cite{beelen-van_dooren}. An implementation of those algorithms has been written
in Julia \cite{bezanson2017julia} by Andreas Varga, it can be found in \url{https://github.com/andreasvarga/MatrixPencils.jl}.

Another approach we may take is to work in an exact ring, one in which numbers are never
approximated, and their values are exact.

Indeed, this work takes a \textit{novel approach} to compute Kronecker's canonical form: we define an algorithm to
compute exactly
Kronecker's canonical form for an arbitrary pencil of matrices. Last, we give an implementation of this
algorithm \cite{Trapani_Computation_of_Kronecker_s} using the open-source computer algebra
system \textbf{SageMath} \cite{sage}.

Even though the focus of this work is not on the applications of pencils of matrices, it may be relevant for a reader to
know they lie in the fields of control systems \cite{824690, 1103983} and signal
processing \cite{1179782}.

\vspace{5mm}

In the first chapter, we recapitulate the basics of numerical analysis, computer algebra and linear algebra required to
understand the following chapters. We also introduce SageMath and use it to show the reader consequences of the given
theorems and the definitions.

In the second chapter, we give a self-contained introduction to the theory of linear pencils of matrices: we define regular
and singular pencils and the conditions for equivalence (between pencils of matrices). Last, we define Kronecker's canonical
form and briefly introduce linear differential-algebraic equations with constant coefficients.

In the third chapter, we prove Kronecker's theorem by defining a procedure to compute Kronecker's canonical form
for an arbitrary pencil of matrices; we also provide the pseudocode for such an algorithm.

In the fourth chapter, we discuss technical details of the SageMath algorithm and
give an approximation of its time complexity; we give a minimal working example and discuss the reasoning behind the way
the test suite has been written.

The fifth chapter concludes this paper and outlines possible improvements for the algorithm.