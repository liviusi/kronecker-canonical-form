\section{Introduction}
This work deals with the problem of \textit{computing precisely} \textbf{Kronecker's canonical form} for an
arbitrary pencil of matrices and defines an algorithm \linebreak to compute it (exactly) using the open-source
computer algebra system \textbf{SageMath}\cite{sage}. We may think of Kronecker's canonical form as a generalization of
Jordan's canonical form. Next, we shall try and make sense of this abstract. Let us start with an example.

Let $A$ be a matrix defined in $\mathbb{C}^{m \times n}$ and consider the linear differential equation
\[
    \dot{x} + Ax = f.
\]
The set of solutions of this equation is characterized by matrix $A$ and, in particular, by Jordan's canonical form
(of matrix $A$).

Next, with $B \in \mathbb{C}^{m \times n}$, take the equation
\begin{gather} \label{intro:1}
    B\dot{x} + Ax = f.
\end{gather}
At this point, we can define what a pencil of matrices is. We call the pair ($A$, $B$) a \textbf{linear pencil
of matrices} or, for the sake of brevity, a pencil of matrices \cite{ikramov}.

To solve \eqref{intro:1}, we shall distinguish two cases.
\begin{cs}
    \case $B$ is nonsingular.

        The set of solutions is characterized by Jordan's normal form of \[-B^{-1}A.\]
    \case $B$ is singular.

        The linear system given by \eqref{intro:1} may contain algebraic equations or, in other words,
        equations which do not contain any derivatives. As an example, take the equation
        \[
            \begin{bmatrix}
                1 & 0 & 1 \\
                0 & 0 & 0 \\
                0 & 1 & 1
            \end{bmatrix}
            \begin{bmatrix}
                \dot{x_1} \\
                \dot{x_2} \\
                \dot{x_3}
            \end{bmatrix} + 
            \begin{bmatrix}
                1 & 0 & 0 \\
                1 & 1 & 1 \\
                0 & 0 & 1
            \end{bmatrix}
            \begin{bmatrix}
                x_1 \\
                x_2 \\
                x_3
            \end{bmatrix}
            = \begin{bmatrix}
                f_1 \\
                f_2 \\
                f_3
            \end{bmatrix}.
        \]
        The linear system contains an algebraic equation
        \begin{equation*}
            \left\{
                \begin{aligned}
                    \dot{x_1} + \dot{x_3} + x_1 &= f_1 \\
                    x_1 + x_2 + x_3 &= f_2 \\
                    \dot{x_2} + \dot{x_3} + x_3 &= f_3
                \end{aligned}
            \right.
        \end{equation*}
        The set of solutions is characterized by \textbf{Kronecker's canonical form} of ($A$, $B$)
        \cite{gantmacher, kunkel-mehrmann}.
\end{cs}
\vspace{5mm}

The problem of computing precisely Kronecker's canonical form is \textit{ill-conditioned}, meaning
that small changes in its input values may cause a large change in the answer. The current state of the art is to
compute a \textbf{Kronecker-like form}, which can be found with numerically stable algorithms
\cite{beelen-van_dooren}. An implementation of those
algorithms has been written in Julia \cite{bezanson2017julia} by Andreas Varga, it can be found in
\url{https://github.com/andreasvarga/MatrixPencils.jl}.

Here, we take \textit{a different approach}; but first, let us investigate whether there exists a way for a
calculator to reason over an ill-conditioned problem.

As an example, take another ill-conditioned problem, i.e. the problem of computing Jordan's canonical form of
an arbitrary matrix: how does SageMath calculate Jordan's canonical form? Computer algebra makes it possible for
a computer to manipulate mathematical expressions involving symbols with no given value, and it is the tool for
this very job: SageMath allows a user to work with symbolic expressions via the symbolic expression ring \textbf{SR}.
\begin{minted}[mathescape,
    numbersep=5pt,
    gobble=2,
    frame=lines,
    framesep=2mm,
    escapeinside=!!]{sage}
!\textcolor{blue}{sage:}! A = matrix(SR, [[-11, -14, 0], [0, -4, 0], [1, 4, -1]])
!\textcolor{blue}{sage:}! J = A.jordan_form()
!\textcolor{blue}{sage:}! J
[-11|  0|  0]
[---+---+---]
[  0| -4|  0]
[---+---+---]
[  0|  0| -1]
!\textcolor{blue}{sage:}! J.base_ring()
Symbolic Ring
\end{minted}

Indeed, this work takes this kind of approach to compute Kronecker's canonical form. We define step-by-step a
procedure to compute exactly Kronecker's canonical form of an arbitrary pencil of matrices and prove its
correctness. Next, we provide an implementation of such a procedure in \cite{trapani-kronecker} making use of
SageMath's support for symbolic expressions, and a suitable test suite to prove it correctly mirrors its
specification. Now, it is a pertinent fact that, while this technique is used to calculate Jordan's canonical
form, \textit{no other works in literature follow this approach for Kronecker's canonical form} (at least, there
are none to our knowledge).

Last, even though the focus of this work is not on the applications of pencils of matrices, it may be relevant
for a reader to know they lie in the fields of control
systems \cite{824690, 1103983} and signal processing \cite{1179782}; further examples can also be found in
\cite[pp. 8-11]{kunkel-mehrmann}, such as electrical networks, multibody systems, chemical
engineering, semidiscretized Stokes equations.

\vspace{5mm}

Now, we shall summarize the structure of this work.

In the first chapter, we recapitulate the basics of numerical analysis, computer algebra and linear algebra required to
understand the following chapters. We also introduce SageMath and use it to show the reader consequences of the given
theorems and the definitions.

In the second chapter, we give a self-contained introduction to the theory of linear pencils of matrices: we
define regular and singular pencils and the conditions for equivalence (between pencils of matrices). Last, we
define Kronecker's canonical form and briefly introduce linear differential-algebraic equations with constant
coefficients.

In the third chapter, we prove Kronecker's theorem by defining a procedure to compute Kronecker's canonical form
for an arbitrary pencil of matrices; we also provide the pseudocode for such an algorithm.

In the fourth chapter, we discuss technical details of the SageMath algorithm and
give an approximation of its time complexity; we give a minimal working example and discuss the reasoning behind the way
the test suite has been written.

The fifth chapter concludes this paper and outlines possible improvements for the algorithm.