\section{Background}
\subsection*{Computer algebra.}
Computers have fundamentally two ways to reason over a mathematical expression: \textbf{numerical computations}, which are performed
using \textit{only numbers} to represent values and \textbf{computer algebra} (or \textbf{symbolic computations}), which - by
contrast - use \textit{both numbers and symbols}.

Hereinafter, given a symbol \textit{x} its numerical value shall be written out as \textit{\(x + \epsilon\)}.

\begin{exmp}
    Let us define the following matrix (made up of both symbols and numbers) M:
    \[
        \begin{bmatrix}
            \sqrt{2}  &   1    \\
                2     & \sqrt{2}
        \end{bmatrix}
    \]
    Consider the matrix \( \tilde{M} \), having the same entries of M taken in numerical form:
    \[
        \begin{bmatrix}
            \sqrt{2} + \epsilon &           1         \\
                2               &  \sqrt{2} + \epsilon
        \end{bmatrix}
    \]
    Computing its determinant gives out \( 2  +2\epsilon\sqrt{2} + \epsilon^2 - 2 \doteq 2 + 2\epsilon\sqrt{2} -2 \neq 0 \).
\end{exmp}

Introducing a small change (e.g. an error) in the input argument may either cause a large or a small change in the result.
We shall now introduce the concept of condition number.

\begin{definition}[Condition number]
    A condition number of a problem measures the sensitivity of the solution to small perturbations in the input data.
    Given a function \(f \):
    \[
        cond(f, x) = \lim_{\epsilon \to 0} \sup \limits_{\norm{\Delta x} \leq \epsilon \norm{x}}
        \dfrac{\norm{f(x+\Delta x) - f(x)}}{\epsilon \norm{f(x)}}
    \]
    Given a problem, if its condition number is low it is said to be \textbf{well-conditioned} (typically \( cond(f, x) \sim 1 \)),
    while a problem with a high condition number is (said to be) \textbf{ill-conditioned} (\( cond(f, x) \gg 1 \)).

    The condition number of a nonsingular matrix A is defined as:
    \[
        \kappa(A) = \norm{A^-1} \norm{A}
    \]
\end{definition}

Let us now investigate what would happen if (rather than using only numbers to compute expressions) symbols are allowed by introducing
a framework which allows to work both with numerical and symbolic computations.
\begin{definition}[Computer algebra system]
    A computer algebra system (CAS) is a mathematics software package that is able to perform \textit{both symbolic and numerical
    mathematical computations}.
\end{definition}

A CAS is usually a \textbf{REPL} expected to support a few functionalities \cite{introcas}:
\begin{itemize}[topsep=0pt, itemsep=0pt, parsep=0pt]
    \item \textbf{Arithmetic}:
        arithmetic over different fields with arbitrary precision.
    \item \textbf{Linear algebra}:
        matrix algebra and knowledge of different operations and properties of matrices
        (i.e. determinants, eigenvalues and eigenvectors).
    \item \textbf{Polynomial manipulation}:
        factorization over different fields, simplification and partial fraction decomposition of rational functions.
    \item \textbf{Trascendental functions}:
        support for trascendental functions and their properties.
    \item \textbf{Calculus}:
        limits, derivatives, integration and expansions of functions.
    \item \textbf{Solving equations}:
        solving systems of linear equations, computing with radicals solutions of polynomials of degree less than five.
    \item \textbf{Programming language}:
        users may implement their own algorithms using a programming language.
\end{itemize}

The CAS chosen for this work is \textbf{SageMath} \cite{sage}.

SageMath is an open source CAS distributed under the terms of the GNU GPLv3 \cite{gpl}.

An attempt to review its functionalities and features shall not be made, but an
example showing it allows for calculations either in symbolic or in inexact ones shall (be provided).
\begin{exmp}
    Take matrix M from Example 2.1:
    \[
        \begin{bmatrix}
            \sqrt{2}  &   1    \\
                2     & \sqrt{2}
        \end{bmatrix}
    \]
    Compare the different result given out by defining M over the \textit{symbolic ring SR} and the \textit{finite-precision ring
    CDF}:
    \begin{minted}[mathescape,
        numbersep=5pt,
        gobble=2,
        frame=lines,
        framesep=2mm]{sage}
        sage: matrix(SR, [[sqrt(2), 1], [2, sqrt(2)]]).det()
        0
        sage: matrix(CDF, [[sqrt(2), 1], [2, sqrt(2)]]).det()
        -3.14018491736755e-16
    \end{minted}
\end{exmp}

\subsection*{Eigenvalues, eigenvectors}

\begin{definition}[Eigenvalue, eigenvector]
    Given a linear transformation \(T\) in a finite-dimensional vector space \(V\) over a field \(F\) into itself and a nonzero
    vector
    \(\vec{v}\), \(\vec{v}\) is an eigenvector of \(T\) if and only if:
    \[ A \vec{u} = \lambda \vec{u} \]
    with \(A\) the matrix representation of \(T\), \(\vec{u}\) the coordinate vector of \(\vec{u}\) and \(\lambda\) a scalar in
    \(F\) known as eigenvalue associated with \(\vec{v}\).
\end{definition}

\begin{remark}
    Note that:
    \[ A \vec{u} = \lambda \vec{u} \iff (\lambda I - A)\vec{u} = 0 \]
    It follows that the eigenvalues of A are the roots of:
    \[ det(\lambda I - A) \]
    which is a polynomial in \(\lambda\) known as the \textbf{characteristic polynomial} \(ch(A)\).
\end{remark}

\subsection*{Jordan canonical form}