\section{Theory and applications of pencils of matrices}
This chapter will introduce the reader to the concept of a linear pencil of matrices and its properties.
\begin{definition}[Linear matrix pencil]
    A linear pencil of matrices is defined as a polynomial with matrix coefficients
    \[
        \Gamma(\lambda) = A + \lambda B
    \]
    with \(\lambda \in \mathbb{C}, A\) and \(B\) \(m \times n\) matrices.
    A linear pencil of matrices may also be called a \textbf{pair of matrices} and, in this thesis, we shall use
    synonymously both terms.
\end{definition}
\begin{definition}[Regular pencil]
    A matrix pair \((A, B)\) is said to be regular if and only if \(A\) and \(B\) are square matrices of the same size and
    the determinant \(det(A + \lambda B)\) is not identically zero.
\end{definition}
\begin{definition}[Singular pencil]
    A matrix pair \(A, B\) is said to be singular if and only if it is not regular.
\end{definition}

\subsection*{Regular pencils.}
Consider the regular pencil of matrices
\[\Gamma(\lambda) = A + \lambda B,\]
let \(F\) be the field the entries of \(A\) and \(B\) belong to and \(r\) the rank of the pencil.

Denote with \(D_{j}(\lambda)\) the greatest common divisor of all minors of order \(j\) of \(\Gamma(\lambda)\)
(with \(j = 1, ..., r\)) and assume without any loss of generality \(D_{j}(\lambda)\) is monic and
\(D_{0}(\lambda) = 1\). Given the sequence,
\begin{gather*}
    D_{r}(\lambda), \
    D_{r-1}(\lambda), \
    ..., \
    D_{1}(\lambda), D_{0}(\lambda)
\end{gather*}
we define the \textbf{invariant polynomials} of the pencil of matrices \(\Gamma(\lambda)\) as the fractions
\begin{gather*}
    i_{1}(\lambda) = \dfrac{D_{r}(\lambda)}{D_{r-1}(\lambda)}, \
    i_{2}(\lambda) = \dfrac{D_{r-1}(\lambda)}{D_{r-2}(\lambda)}, \
    ..., \
    i_{r}(\lambda) = D_{1}(\lambda).
\end{gather*}

We can now write the expansion of the invariant polynomials into irreducible factors in \(F\) as
\begin{gather*}
    i_{1}(\lambda) = \prod_{i=1}^{k}p_{i}(\lambda)^{\alpha_{1, i}}, \
    i_{2}(\lambda) = \prod_{i=1}^{k}p_{i}(\lambda)^{\alpha_{2, i}}, \
    ... \\
    i_{r}(\lambda) = \prod_{i=1}^{k}p_{i}(\lambda)^{\alpha_{r, i}},
\end{gather*}
with \(p_{i}\) an irreducible polynomial appearing in the expansion.

We define the \textbf{elementary divisors} \(e_{i}\) of the pencil of matrices \(\Gamma(\lambda)\) all the polynomials
\(p_{i}(\lambda)^{\alpha_{j, i}}\) (with \(j = 1, ..., r\)) that are not equal to one.

A similar procedure may be defined for the pencil of matrices
\[
    \Theta(\lambda) = \mu A + \lambda B
\]
leading to polynomials in two variables \((\mu, \lambda)\). Clearly, having \(\mu = 1\) would lead to obtaining the elementary
divisors of \(\Gamma(\lambda)\); however, for each elementary divisor of degree \(q\) we have
\[
    e_{i}(\mu, \lambda) = \mu^q e_{i}(\dfrac{\lambda}{\mu}),
\]
and, with this technique, it is possible to generate all the elementary divisors of \(\Theta(\lambda)\) except for
those of the form \(\mu^q\), which are called \textbf{elementary infinite divisors} of the pencil of matrices \(\Theta(\lambda)\).

\begin{remark}
    A regular pencil of matrices \(\Gamma(\lambda) = A + \lambda B\) has elementary infinite divisors if and only if
    \(det(B) = 0\).
\end{remark}

We can now give a result on the equivalence of regular pencils.

\begin{theorem}[Equivalence of regular pencils of matrices]
    Two regular matrix pairs \((A, B)\), \((A_{1}, B_{1})\) are called equivalent if and only if they have the same finite and
    infinite elementary divisors.
\end{theorem}

\subsection*{Singular pencils.}

\subsection*{Kronecker canonical form.}

\subsection*{Fundamental applications.}