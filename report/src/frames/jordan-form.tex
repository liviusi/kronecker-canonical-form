\begin{frame}
	\frametitle{Forma canonica di Jordan}
		\begin{definition}
			Una matrice $J$ diagonale a blocchi viene detta \emph{matrice di Jordan} se e solo se
			ogni blocco lungo la diagonale \`e quadrato ed \`e del tipo
			\[
				\begin{bmatrix}
					\lambda     &    1         &    0     &   \cdots   &    0    \\
						0       &    \lambda   &    1     &   \cdots   &    0    \\
					\vdots      &    \vdots    &  \vdots  &   \ddots   & \vdots  \\
						0       &       0      &     0    &  \lambda   &    1    \\
						0       &       0      &     0    &     0      & \lambda
				\end{bmatrix}.
			\]
		\end{definition}
		\onslide<2-> Ogni matrice \`e simile a una matrice di Jordan.
\end{frame}


\begin{frame}
	\frametitle{Forma canonica di Jordan}
	\framesubtitle{Stabilit\`a della trasformazione}
	Data una matrice quadrata $A$, la matrice di trasformazione P in
	\[A = P^{-1}JP\]
	\`e malcondizionata se $A$ ha un autovalore difettivo o quasi difettivo.
\end{frame}