\begin{frame}[fragile]
	\frametitle{Calcolo esatto della forma canonica di Kronecker}
	\framesubtitle{Snippet di codice}
	Assumiamo di aver definito la coppia di matrici in forma di Kronecker ($A$, $B$) con
	\begin{align*}
		A &=
		\begin{bmatrix}
			0 & 1 & 0 & 0 \\
			0 & 0 & 1 & 0 \\
			0 & 0 & 0 & 1 \\
			& & & & 1 & 0 \\
			& & & & 0 & 1 \\
			& & & & & & 42
		\end{bmatrix}, &
		B &=
		\begin{bmatrix}
			1 & 0 & 0 & 0 \\
			0 & 1 & 0 & 0 \\
			0 & 0 & 1 & 0 \\
			& & & & 0 & 1 \\
			& & & & 0 & 0 \\
			& & & & & & 1
		\end{bmatrix}.
	\end{align*}
	Importiamo i moduli necessari.
	\begin{minted}[numbersep=5pt,
		frame=lines,
		framesep=2mm]{python}
import sage.all as sa
from kcf import kcf_sage as kcf
	\end{minted}
\end{frame}


\begin{frame}[fragile]
	\frametitle{Calcolo esatto della forma canonica di Kronecker}
	\framesubtitle{Snippet di codice}
	Definiamo un metodo per calcolare una matrice casuale invertibile.
	\begin{minted}[numbersep=5pt,
		frame=lines,
		framesep=2mm]{python}
def random_invertible_matrix(n: int):
	while True:
		M = sa.random_matrix(sa.ZZ, n, n)
		if not (M.is_singular()):
			return M.change_ring(sa.SR)

D = random_invertible_matrix(A.nrows())
C = random_invertible_matrix(A.ncols())
	\end{minted}
\end{frame}


\begin{frame}[fragile]
	\frametitle{Calcolo esatto della forma canonica di Kronecker}
	\framesubtitle{Snippet di codice}
	Calcoliamo la forma canonica di Kronecker della coppia di matrici ($A$, $B$) dopo aver
	cambiato la loro base.
	\begin{minted}[numbersep=5pt,
		frame=lines,
		framesep=2mm]{python}
A = D.inverse() * A * C
B = D.inverse() * B * C

(L, R), (KCF_A, KCF_B) = kcf.kronecker_canonical_form(
	A, B, transformation=True)
assert ((L*A*R - KCF_A).is_zero()
		and (L*B*R - KCF_B).is_zero()
		and not L.is_singular()
		and not R.is_singular())
print(f'{kcf.stringify_pencil(KCF_A, KCF_B)}')
\end{minted}
\end{frame}

\begin{frame}
	\frametitle{Calcolo esatto della forma canonica di Kronecker}
	\framesubtitle{Risultati}
	I risultati ottenuti da questa tesi sono:
	\begin{enumerate}
		\item aver esteso la letteratura riguardo al calcolo della forma canonica di Kronecker,
		\item descritto e dimostrato un algoritmo che la calcola esattamente,
		\item implementato e testato tale algoritmo.
	\end{enumerate}
\end{frame}