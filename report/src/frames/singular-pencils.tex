\begin{frame}
	\frametitle{Pencil singolari}
	\framesubtitle{Determinare un polinomio di grado minimo nel kernel destro}
	Sia \(\Gamma(\lambda)=A+\lambda B\) un pencil singolare di matrici.

	Iterando sul valore di $k = 0$ con passo $1$, consideriamo la famiglia di matrici
	$M_k^{(A, B)}$ di dimensione $(k+2) \times (k+1)$
	\begin{align*}
		M_{0}^{(A, B)} &=
			\left[\begin{smallmatrix}
				A \\
				B
			\end{smallmatrix}\right],
		& M_{1}^{(A, B)} &=
			\left[\begin{smallmatrix}
				A & 0 \\
				B & A \\
				0 & B
			\end{smallmatrix}\right],
		& M_{k}^{(A, B)} &=
			\left[\begin{smallmatrix}
				A & 0 & \hdots &    0   \\
				B & A &        & \vdots \\
				0 & B & \ddots & \\
				\vdots & \vdots & \ddots & A \\
				0      &    0   & \hdots & B
			\end{smallmatrix}\right].
	\end{align*}
	\onslide<2-> La matrice di base del kernel destro di $M_k$, divisa in blocchi di
	dimensione $k+1$, identifica un polinomio di grado $k$ nel kernel destro di
	$\Gamma(\lambda)$.
\end{frame}


\begin{frame}
	\frametitle{Pencil singolari}
	\framesubtitle{Indice minimo per le colonne}
	Assumiamo senza perdita
	di generalit\`a che il suo rango $r$ sia minore del numero di colonne $n$. Allora,
	esistono soluzioni non banali $\vb{x_1}(\lambda), ..., \vb{x_r}(\lambda)$ dell'equazione
	\[
		(A+\lambda B)\vb{x} = 0.
	\]
	\onslide<2-> Il numero massimo di soluzioni linearmente indipendenti \`e $n-r$. Scegliamo
	sempre il polinomio di grado minimo e, iterando, otteniamo la sequenza
	\[
		\epsilon_{1} \leq \epsilon_{2} \leq ... \leq \epsilon_{p}.
	\]
	Definiamo i termini $\epsilon_i$ \emph{indici minimi per le colonne}.
\end{frame}


\begin{frame}
	\frametitle{Pencil singolari}
	\framesubtitle{Teorema di riduzione}
	\begin{theorem}[Teorema di riduzione]
		Se il polinomio di grado minimo nel kernel destro ha grado $\epsilon > 0$, allora esistono
		$P$, $Q$ matrici quadrate costanti invertibili tali che
		\[
			P\Gamma(\lambda)Q=\begin{bmatrix}
				L_{\epsilon} & 0 \\
				0 & \widehat{A} + \lambda \widehat{B}
			\end{bmatrix}.
		\]
	\end{theorem}
	\onslide<2-> Il pencil ottenuto $\widehat{\Gamma}(\lambda) = \widehat{A} + \lambda \widehat{B}$
	non ha polinomi di grado inferiore a $\epsilon$ nel proprio kernel destro.
\end{frame}