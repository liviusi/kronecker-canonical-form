\begin{frame}
	\frametitle{Equazioni differenziali lineari a coefficienti costanti}
	\begin{itemize}
		\item Consideriamo le equazioni
		differenziali del tipo
		\begin{align*}
			\dot{x}(t) + Ax(t) &= f(t),& A &\in \mathbb{C}^{m \times n}.
		\end{align*}
		\item \onslide<2-> Le soluzioni sono caratterizzate dalla forma canonica
		di Jordan della matrice A.

		\item \onslide<3-> Generalizziamo.
	
		Introduciamo una matrice $B \in \mathbb{C}^{m \times n}$. Dunque, consideriamo
		le equazioni del tipo
		\[
			B\dot{x}(t) + Ax(t) = f(t).
		\]
	\end{itemize}
\end{frame}

\begin{frame}
	\frametitle{Equazioni differenziali algebriche lineari a coefficienti costanti}
	Distinguiamo due casi.
	\begin{enumerate}
		\item $B$ non \`e singolare.
		\begin{itemize}
			\item \onslide<2-> Le soluzioni sono caratterizzate dalla forma
			di Jordan della matrice $-B^{-1}A$.
		\end{itemize}
		\item \onslide<3-> $B$ \`e singolare.
		\begin{itemize}
			\item \onslide<4-> Essendo B una matrice singolare, alcune equazioni nel sistema lineare potrebbero
			essere algebriche o, in altre parole, non contenere derivate.
			\item \onslide<5-> Le soluzioni sono caratterizzate dalla \emph{forma canonica di
			Kronecker} della coppia di matrici ($A$, $B$) (detta anche \emph{linear pencil} o,
			per brevit\`a, \emph{pencil}).
		\end{itemize}
	\end{enumerate}
	
\end{frame}