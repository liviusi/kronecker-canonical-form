\begin{frame}
	\frametitle{Pencil regolari}
	\framesubtitle{Polinomi invarianti}
	Assumiamo che il pencil $\Gamma(\lambda)=A+\lambda B$ abbia rango $r$.

	Sia \(D_{j}(\lambda)\) il massimo comun divisore dei minori di ordine $j$
	di $\Gamma(\lambda)$ (con $j=1, ..., r$). Assumiamo senza perdita di generalit\`a
	$D_r(\lambda)$ abbia coefficiente 1 e $D_0(\lambda)=1$.
	Definiamo \emph{polinomi invarianti} del
	pencil \(\Gamma(\lambda)\) le frazioni
	\begin{gather*}
		i_{1}(\lambda) = \dfrac{D_{r}(\lambda)}{D_{r-1}(\lambda)}, \
		i_{2}(\lambda) = \dfrac{D_{r-1}(\lambda)}{D_{r-2}(\lambda)}, \
		..., \
		i_{r}(\lambda) = D_{1}(\lambda).
	\end{gather*}
	Siano $p_i$ polinomi irriducibili. Possiamo scrivere l'espansione dei polinomi invarianti
	in fattori irriducibili come
	\begin{gather*}
		i_{1}(\lambda) = \prod_{i=1}^{k}p_{i}(\lambda)^{\alpha_{1, i}}, \
		i_{2}(\lambda) = \prod_{i=1}^{k}p_{i}(\lambda)^{\alpha_{2, i}}, \
		... \\
		i_{r}(\lambda) = \prod_{i=1}^{k}p_{i}(\lambda)^{\alpha_{r, i}}.
	\end{gather*}
\end{frame}


\begin{frame}
	\frametitle{Pencil regolari}
	\framesubtitle{Divisori elementari finiti, divisori elementari infiniti}
	\begin{definition}[Divisori elementari finiti]
		Definiamo divisori elementari finiti di \(\Gamma(\lambda)\) tutti i
		polinomi $p_i(\lambda)$ diversi da 1.
	\end{definition}
	\begin{definition}[Divisori elementari infiniti]
		Definiamo
		divisori elementari infiniti di \(\Gamma(\lambda)\) i divisori elementari finiti
		del pencil $\lambda A + B$.
	\end{definition}
\end{frame}


\begin{frame}
	\frametitle{Pencil lineari regolari}
	\begin{theorem}[Forma canonica di Kronecker - pencil regolari]
		Ogni pencil regolare pu\`o essere ridotto a una matrice del tipo
		\[\begin{bmatrix}
			N^{(u_1)} \\
			& N^{(u_2)} \\
			&& \ddots \\
			&&& N^{(u_s)} \\
			&&&& G + \lambda I
		\end{bmatrix}.\]
	\end{theorem}
\end{frame}