\begin{frame}
	\frametitle{Calcolo simbolico}
	\begin{itemize}
		\item Il \emph{calcolo simbolico} (o \emph{computer algebra}) - utilizzando sia
		\emph{variabili} sia \emph{valori numerici} - permette il calcolo esatto di
		espressioni matematiche.

		\item \onslide<2-> Il sistema per il calcolo simbolico scelto \`e \emph{SageMath}.
	\end{itemize}
\end{frame}

\begin{frame}[fragile]
	\frametitle{Calcolo simbolico}
	\framesubtitle{Esempio: calcolo del determinante di una matrice}
	Calcoliamo il determinante della matrice $A$, con
	\[
		A = \begin{bmatrix}
			\sqrt{3} & 1 \\
			3 & \sqrt{3}
		\end{bmatrix}.
	\]
	\begin{minted}[fontsize=\footnotesize,
		frame=lines,
		framesep=2mm,
		escapeinside=!!]{sage}
!\textcolor{blue}{sage:}! A = matrix(SR, [[sqrt(3), 1], [3, sqrt(3)]])
!\textcolor{blue}{sage:}! A.det().is_zero()
True
!\textcolor{blue}{sage:}! A.change_ring(CDF).det().is_zero()
False
	\end{minted}
\end{frame}