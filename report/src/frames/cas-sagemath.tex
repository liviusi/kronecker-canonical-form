\begin{frame}
	\frametitle{Calcolo simbolico}
	\begin{itemize}
		\item Il \emph{calcolo simbolico} (o \emph{computer algebra}) - utilizzando sia
		\emph{variabili} sia \emph{valori numerici} - permette il calcolo esatto di
		espressioni matematiche.

		\item \onslide<2-> Il sistema per il calcolo simbolico scelto \`e \emph{SageMath}.
	\end{itemize}
\end{frame}

\begin{frame}
	\frametitle{Esempio: calcolo del determinante}
	Calcoliamo il determinante della matrice $A$, con
	\[
		A = \begin{bmatrix}
			\sqrt{3} & 1 \\
			3 & \sqrt{3}
		\end{bmatrix}.
	\]
	Confrontiamo i risultati ottenuti definendola
	inizialmente sull'anello \mintinline{sage}{CDF} (\emph{Complex Double Field})
	e, in seguito, sull'anello simbolico \mintinline{sage}{SR} (\emph{Symbolic Ring}).
\end{frame}

\begin{frame}[fragile]
	\frametitle{Esempio: calcolo del determinante}
		\begin{minted}[fontsize=\footnotesize,
			frame=lines,
			framesep=2mm,
			escapeinside=!!]{sage}
!\textcolor{blue}{sage:}! A = matrix(SR, [[sqrt(3), 1], [3, sqrt(3)]])
!\textcolor{blue}{sage:}! A.det().is_zero()
True
!\textcolor{blue}{sage:}! A.change_ring(CDF).det().is_zero()
False
		\end{minted}
\end{frame}

\begin{frame}
	\frametitle{Modello di calcolo}
	\begin{itemize}
		\item Useremo l'anello simbolico \mintinline{sage}{SR} per il calcolo esatto.
		\item \onslide<2-> SageMath mette a disposizione due metodi che, per il nostro caso
		d'uso, sono particolarmente utili:
		\begin{itemize}
			\item \onslide<3-> \mintinline{sage}{is_zero}
			per verificare che una espressione sia esattamente uguale a zero;
			\item \onslide<4-> \mintinline{sage}{jordan_form}
			per calcolare esattamente la forma canonica di Jordan di una matrice.
		\end{itemize}
	\end{itemize}
\end{frame}