\begin{frame}
	\frametitle{Forma canonica di Kronecker}
	\begin{theorem}[Forma canonica di Kronecker]
		Un pencil arbitrario di matrici ($A$, $B$)
		\`e strettamente equivalente alla matrice diagonale a blocchi
		\[
			\begin{bmatrix}
				O^{(h, g)} \\
				& L \\
				&& L^T \\
				&&& N \\
				&&&& G+\lambda I
			\end{bmatrix}.
		\]
	\end{theorem}
\end{frame}


\begin{frame}
	\frametitle{Forma canonica di Kronecker}
	\begin{itemize}
		\item $h$, $g$ sono il numero massimo di soluzioni costanti e indipendenti delle
		equazioni
		\begin{align*}
			(A + \lambda B)\vb{x} &= 0, & (A^T + \lambda B^T)\vb{y} &= 0.
		\end{align*}
		\item \onslide<2-> I blocchi sono del tipo:
		\begin{align*}
			L &= \left[\begin{smallmatrix}
				L_{\epsilon_{h+1}} \\
				& L_{\epsilon_{h+2}} \\
				& & \ddots \\
				& & & L_{\epsilon_{p}}
			\end{smallmatrix}\right], &
			L_{i}^{(i, i+1)} &= \left[\begin{smallmatrix}
				\lambda        &      1     &       0      &     \ldots       &    0       &    0   \\
				0              & \lambda    &       1      &                  & \vdots     & \vdots \\
				\vdots         & \vdots     &     & \ddots    &      &            &        \\
				0              &      0     &              &                  & \lambda    &    1   
			\end{smallmatrix}\right]
		\end{align*}
	\end{itemize}
\end{frame}


\begin{frame}
	\frametitle{Forma canonica di Kronecker}
	\begin{align*}
		L^T &= \left[\begin{smallmatrix}
			L_{\eta_{h+1}}^T \\
			& L_{\eta_{h+2}}^T \\
			& & \ddots \\
			& & & L_{\eta_{q}^T}
		\end{smallmatrix}\right], &
		N &= \left[\begin{smallmatrix}
			N^{(u_{1})} \\
			& N^{(u_{2})} \\ 
			& & \ddots \\
			& & & N^{(u_{s})}.
		\end{smallmatrix}\right]
	\end{align*}
	\begin{itemize}
		\item \onslide<2-> I blocchi $N^{u}$ sono del tipo
		\[N^{(u)} = I^{(u)} + \lambda H^{(u)}.\]
		\item \onslide<3-> $G$ \`e una matrice di Jordan.
	\end{itemize}
\end{frame}


\begin{frame}
	\frametitle{Pencil di matrici}
	Distinguiamo due tipi di pencil di matrici:
	\begin{definition}[Pencil lineare regolare]
		Un pencil di matrici ($A$, $B$) viene definito regolare se e solo se
		$A$ e $B$ sono matrici quadrate della stessa dimensione e il determinante
		$det(A+\lambda B)$ non \`e identicamente zero.
	\end{definition}
	\begin{definition}[Pencil lineare singolare]
			Un pencil di matrici non regolare viene definito singolare.
	\end{definition}
\end{frame}