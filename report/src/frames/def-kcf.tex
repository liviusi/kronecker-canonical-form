\begin{frame}
	\frametitle{Forma canonica di Kronecker}
	\begin{theorem}[Forma canonica di Kronecker]
		Sia \(\Gamma(\lambda) = A + \lambda B\) un pencil di matrici arbitrarie con
		$A$, $B$ di dimensione \(m \times n\). Esistono $P$, $Q$
		matrici quadrate costanti invertibili delle dimensioni appropriate tali che
		\[
			P\Gamma(\lambda)Q = \begin{bmatrix}
				O^{(h, g)} \\
				& L \\
				&& L^T \\
				&&& N \\
				&&&& G+\lambda I
			\end{bmatrix}.
		\]
	\end{theorem}
\end{frame}


\begin{frame}
	\frametitle{Forma canonica di Kronecker}
	\framesubtitle{Descrizione dei blocchi}
	\begin{itemize}
		\item $h$, $g$ sono il numero massimo di soluzioni costanti e indipendenti delle
		equazioni
		\begin{align*}
			(A + \lambda B)\vb{x} &= 0, & (A^T + \lambda B^T)\vb{y} &= 0.
		\end{align*}
		\item \onslide<2-> I blocchi sono del tipo:
		\begin{align*}
			L &= diag\{L_{\epsilon_{h+1}}, L_{\epsilon_{h+2}}, ..., L_{\epsilon_{p}}\},
			&
			L_{i}^{(i, i+1)} &= \begin{bmatrix}
				\lambda & 1 \\
				& \ddots \\
				& & \lambda & 1
			\end{bmatrix}
		\end{align*}
	\end{itemize}
\end{frame}


\begin{frame}
	\frametitle{Forma canonica di Kronecker}
	\framesubtitle{Descrizione dei blocchi}
	\begin{align*}
		L^T &= diag\{L_{\eta_{h+1}}^T, L_{\eta_{h+2}}^T, ..., L_{\eta_{q}^T}\},
		&
		N &= diag\{N^{(u_{1})}, N^{(u_{2})} ..., N^{(u_{s})}\}
	\end{align*}
	\begin{itemize}
		\item \onslide<2-> I blocchi $N^{u}$ sono del tipo
		\begin{align*}
			N^{(u)} &= I^{(u)} + \lambda H^{(u)}, &
			H^{(i)} &= \begin{bmatrix}
				0 & 1 \\
				& \ddots & \ddots \\
				& & 0 & 1 \\
				& & & 0
			\end{bmatrix}
		\end{align*}
		\item \onslide<3-> $G$ \`e una matrice di Jordan.
	\end{itemize}
\end{frame}


\begin{frame}
	\frametitle{Pencil di matrici}
	Distinguiamo due tipi di pencil di matrici:
	\begin{definition}[Pencil lineare regolare]
		Un pencil di matrici ($A$, $B$) viene definito \emph{regolare} se e solo se
		$A$ e $B$ sono matrici quadrate della stessa dimensione e il determinante
		$det(A+\lambda B)$ non \`e identicamente zero.
	\end{definition}
	\begin{definition}[Pencil lineare singolare]
			Un pencil di matrici non regolare viene definito \emph{singolare}.
	\end{definition}
\end{frame}